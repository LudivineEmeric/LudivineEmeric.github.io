\Chapter{Le champ magnétique}

\nivpre{CPGE}{
 \begin{itemize}
  \item électrostatique (équations de Maxwell et champ électrostatique)
  \item théorèmes d'Ostrogradski et de Stokes
  \item coordonnées cylindriques
  \item définition de l'intensité
 \end{itemize}
}

\lemessage{.}

\biblio{}

\section*{Introduction}
Grandeur vectorielle \\
applications au quotidien : aimants \\
champ magnétique terrestre mesuré par boussoles \\

\section{Propriétés du champ magnétique}
\subsection{Sources}
1820 Oersted, physicien dannois, expérience : aiguille métallique à proximité d'un circuit électrique \\
une interprétation possible : source de champ magnétique = déplacement de charges, ie courant électrique \\
ordres de grandeurs... \\

\subsection{Flux du champ $\vec{B}$}
flux à travers surface fermée est nul \\
lignes de champ sont fermées \\
\subsection{Circulation du champ $\vec{B}$}
théorème d'Ampère \\

\section{Expression du champ magnétique}
\subsection{Symétries}
de la distribution de courant \\
à partir de la force de Lorentz \\
\subsection{Invariances}
\subsection{Application à un câble cylindrique}

\section{Induction électromagnétique}
schéma mouvement d'un aimant dans une spire : production d'un tension \\
Loi de Faraday \\
Loi de Lenz \\
Applications : microphone, surtout haut-parleur \\

\section*{Conclusion}
Champ magnétique terrestre : dipôle magnétique (sud au nord géographique, angle $11.5^o$ par rapport à l'axe de rotation de la Terre) \\
moteurs \\



\begin{remarques} \begin{itemize} 
\item $\div \vec{B} = 0$ car il n'y a pas de monopole magnétique ; si ça existait : th de Gauss, équivalent du champ électrostatique radial Coulomb
\item créer un champ magnétique qui simule un champ magnétique : solénoide semi-infini (très grande longueur, regardée à une distance intermédiaire) $\rightarrow$ champ radial
\item différence entre force de Lorentz et force de Laplace : la force de Lorentz ne travaille pas mais la force de Laplace peut travailler car force appliquée par les charges au conducteur, perpendiculaire au courant mais le déplacement est arbitraire   
\item cas où il y a une influence des charges sur le champ magnétique : charges en mouvement ; il faut aussi considérer leur vitesse pour connaître influence
\item cas d'une bobine carrée
\item cas d'une bobine torique
\item calcul champ spire sur l'axe
\item solénoïde
\item H historiquement a été introduit en premier
\item il faut absolument parler de l'aimant
\item penser aux analogies dipôle électrique/dipôle magnétique
\end{itemize} \end{remarques}
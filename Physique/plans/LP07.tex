\Chapter{Transitions de phase}

\nivpre{L3}{
 \begin{itemize}
  \item potentiels thermodynamiques
  \item électromagnétisme dans les milieux
 \end{itemize}
}

\lemessage{Message important à faire passer lors de la leçon.}

\biblio{}

\section*{Introduction}
définition phase

\section{Transition liquide-vapeur d'un corps pur}
\subsection{Approche phénoménologique}

Description (P,T), (P,V) \\
Point critique, triple

\begin{ecran}
	Diagrammes thermodynamiques de l'eau
\end{ecran}

\subsection{Étude thermodynamique}
\subsection{Discontinuité et relation de Clapeyron}
\subsection{Mise en œuvre expérimentale}

\section{Transition paramagnétique-ferromagnétique}
\subsection{Modélisation}
\subsection{Minimisation du potentiel}



\begin{experience}
	Chauffer diazote liquide, peser et déterminer la masse perdue par vaporisation ; faire sans chauffer, puis en chauffant
\end{experience}



\section*{Conclusion}

H=f(M) pour différent T, quand T diminue en-dessous de température de Curie : bifurcation fourche


\begin{remarques} \begin{itemize}
\item point critique : opalescence
\item eau : liquide-vapeur pente (P,T) négative 
\item Landau : discussion de symétries, brisure de symétrie transition para-ferro, on privilégie un certain axe ; la phase la plus ordonnée est ferro (ordre magnétique)
\item liquide-vapeur : pas de brisure de symétrie car au-delà du point critique, passe continûment de l'un à l'autre
\item autres transitions de phase : métal supraconductrice vers conducteur (transition du premier ordre ?), condensat de Bose-Einstein superfluide, variétés allotropiques, cristaux liquides
\item état métastable : la création de surface entre deux phases a un coût, le changement d'état est retardé, un système dans un tel état peut être amené vers l'état stable à partir d'une petite perturbation (exemple : eau congélateur, sortir au bon moment, une pichenette et ça gèle instantanément)
\item autre exemple métastable : chevaux hiver 1942 Leningrad, lac Ladoga, congelés dans l'eau
\item liquide-vapeur métastable : nucléation dans casserole quand on fait chauffer l'eau : les bulles viennent d'endroits particuliers, nécessite défauts. Pression diminue tant qu’aucune bulle de vapeur n’apparaît pas. Pression peut être négative dans le liquide si pas de défaut paroi : la cohésion des molécules entre elles et avec les parois  permet d’exercer cette pression négative (terme $a/V^2$ dans van der waals pour fluide) (\href{https://www-liphy.univ-grenoble-alpes.fr/pagesperso/marmottant/Publications_files/ArticleFinalCouverturePlusArticle.pdf}{lien pour sources}). Les pressions négatives observées dans de l’eau peuvent atteindre plusieurs centaines de fois la pression atmosphérique ! Cavitation : défaut provoque la nucléation d'une bulle
\item solide-liquide métastable : brouillard givrant (eau en surfusion)
\item le diamant est métastable
\item chambre à bulles : hydrogène liquide maintenu dans un état surchauffé, champ magnétique, trajectoire particule courbée, matérialisée par formation d'une trainée de bulles (Prix Nobel 1960)
\item chambre à brouillard : inverse, trainée de condensation (Prix Nobel 1927, ancêtre ?) pour l'étude de particules radioactives, étude  de produits de réactions nucléaires, étude d'interactions... Charles Wilson l'a inventé en essayant de comprendre pourquoi les nuages se forment. Première méthode d'imagerie
\item cristaux liquides : transition smectique-nématique... vaste sujet chimique
\item le fer n'est pas un bon exemple de transition para-ferro car il y a aussi un changement de variété allotropique vers la température de Curie
\item brisure spontanée de symétrie généralisée par chapeau mexicain : modèle de Higgs
\end{itemize} \end{remarques}
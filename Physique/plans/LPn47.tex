\Chapter{Adaptation d'impédance}

\nivpre{CPGE}{
 \begin{itemize}
  \item équation de d'Alembert
   \item modélisation du câble coaxial
   \item AO
 \end{itemize}
}

\lemessage{Partout, c'est la vie.}

\biblio{}

\section*{Introduction}

Permettre un transfert de puissance optimal \\
Couche anti-reflet, cornets, cuivre (instrument), picots en salle insonorisée ou au plafond de salles... \\

\marginpar{réf. \cite{Berkeley_ondes}, \cite{Thibierge_ondes}}

Voir Berkeley, Landau et Feynman \\

\section{Câble coaxial : réflexion sur une impédance terminale}
\subsection{Rappel}
équations des télégraphes sans pertes \\
\subsection{Réflexion sur une résistance terminale}
Résistance placée à la suite du câble \\
Détermination des coefficients de réflexion en amplitude \\
$R=Z_C$ : réflexion nulle, c'est ce qu'on appelle adaptation d'impédance \\
intérêt du montage suiveur \\
\section{Application pratique en acoustique : l'échographie}
\subsection{Interface air-muscle}
intensité réfléchie, intensité transmise \\
$R=\left( \frac{Z_2 - Z_1}{Z_2 + Z_1} \right)^2 $ \\
$T=4 \frac{Z_2 Z_1}{\left( Z_2 + Z_1\right)^2} $  \\
$Z_\mathrm{air}=444 kg.m^2.s^{-1}$ et $Z_\mathrm{muscle}=1,7 . 10^6 kg.m^2.s^{-1}$ \\
$T=10^{-3}$ très faible

\subsection{Gel entre les deux}
a priori, impédance du gel est entre les 2 \\
Attention, le calcul est bizarre : ne pas se fier au td d'hydro \\
en supposant qu'il n 'y a pas de réflexion :
condition d'adaptation d'impédance $\rightarrow$ $\frac{Z_g-Z_a}{Z_g+Z_a}=\frac{Z_g-Z_m}{Z_g-Z_m} e^{-2 i k_g e}$ \\
les impédances ne sont pas complexes car équation vraie que pour certains e \\

conditions aux limites $\rightarrow$  $2 k_g e=2n\pi$ ou $(2n+1) \pi$, paire ou impaire $\rightarrow$ solution intéressante : impaire \\
$Z_g=\sqrt{Z_a Z_m}$ \\
$Z_g=2.7 .10^4 kg.m^{-2}.s^{-1}$

\section*{Conclusion}
maximiser intensité transmise \\

\begin{remarques} \begin{itemize} 
\item exo de l'échographie est schématique : en vrai ce n'est pas de l'air, c'est l'émetteur
\item pourquoi il n'y a pas de réflexion dans l'air ? 
\item Fabry-Pérot, couche anti-reflet
\item suiveur : désadapte l'impédance
\item électronique : on parle d'adaptation d'impédance pour maximiser le rendement
\item quand on cherche à annuler la réflexion, on maximise la puissance transmise
\item impédance d'un dipôle : extensif
\item impédance d'un milieu : intensif 
\item on les traite différemment : ça dépend de la dimension à laquelle on les regarde
\item impédance caractéristique
\item cornet : (ex ondes centimétriques) fait varier continûment l'impédance
\item impédance : lien entre une force cinétique et une force de rappel ; oscillation de l'énergie par passage de forme cinétique à potentielle 
\item coefficients de Fresnel
\item $Z=\frac{"force"}{"déplacement"}$
\item conditions aux limites : dépend du rapport entre les impédances, on peut donc tout ramener aux impédances, ça rebondit
\item cas des dipôles : traité comme ponctuel
\item trompette, réflexion du son sur un mur ; boîte d'oeuf : tout est absorbé
\item déferlement de vagues ?
\item 
\end{itemize} \end{remarques}
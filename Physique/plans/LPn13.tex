\Chapter{Effet Doppler et applications}

\nivpre{Licence}{
 \begin{itemize}
  \item cinématique relativiste
   \item mécanique classique
   \item optique ondulatoire
 \end{itemize}
}

\lemessage{Tout type d'ondes. Sirènes qui s'approche/s'éloigne, radar voiture et redshift des étoiles qui s'éloignent.}

\biblio{}

\section*{Introduction}
prononcer Christiane Doppler (autrichien) \\
fréquence du son détermine sa hauteur (grave/aigu) \\
émetteur en mouvement : son différent \\
on le voit avec sirène qui se déplace \\
parler de monsieur Doppler, proposer mais faux car pas rélativité prise en compte \\
on va le traiter quantitativement \\


%\marginpar{réf. \cite{ToutenunPC_2016},\cite{hecht},\cite{huard_pola}}

\section{Effet Doppler classique}
$S$ : source d'ondes à la fréquence $\nu_0$, vitesse $v_S$ \\
$R$ : récepteur, reçoit à $\nu_1$, vitesse $v_R$ \\
$c$ : célérité des ondes \\
$\vec{u}$ : vecteur unitaire entre S et R

\subsection{Source mobile, récepteur fixe}
première impulsion : $t_1'=t_1 + \frac{d_1}{c}$ \\
deuxième impulsion : $t_2'=t_2 + \frac{d_2}{c}$ \\
$T_0 = t_2 - t_1$ et $T'=t_2'-t_1'$ : $T'=T_0 + \frac{d_2-d_1}{c}$ \\
calcul ... \\
hyp : $\vec{u}$ ne bouge pas trop \\

$f'=f_0 \frac{1}{1-\frac{\vec{v}_S \cdot \vec{u}}{c}}$ \\

Source s'éloigne de R : $f'<f_0$ : le son est plus grave \\
Source s'approche de R : $f'>f_0$ : le son est plus aigu \\

animation internet : un véhicule émet des ondes en se déplaçant \\
à l'avant du véhicule : fronts d'onde sont plus resserrés vers l'avant (aigu, longueur d'onde plus faible), plus éloignés vers l'arrière (grave)

\subsection{Source fixe, récepteur mobile}

Astuce : changement de référentiel \\
$f'=f_0 \left[ 1-\frac{\vec{v}_R \cdot \vec{u}}{c} \right]$ \\

\subsection{Formulation générale}
$f'=f_0 \frac{c-\vec{v}_R \cdot \vec{u}}{c-\vec{v}_S \cdot \vec{u}}$ \\
décalage Doppler $\frac{\Delta f}{f_0}$\\

vitesse du son, avion supersonique, formation d'un cône de front d'onde, énergie s'accumule, onde de choc \\

\section{Effet Doppler relativiste} 

transformée de Lorentz \\
étudier $\omega$ et $\vec{k}$
$f'=f_0 \sqrt{\frac{c-v}{c+v}}$ \\ 
Formule de Doppler-Fizeau : $f'=f_0 \gamma \left( 1-\frac{\vec{v}}{c} \cdot \vec{u} \right)$ \\
si la vitesse est perpendiculaire à l'axe $\vec{u}$, $f'=\gamma f_0 > f_0$ 



\section{Applications}
\subsection{Mesure de vitesse : radar}
détection hétérodyne  \\
signal reçu : $\omega_r$ \\
signal émis : $\omega_s$ \\
multiplieur : on obtient $\omega_r-\omega_s$ et $\omega_r+\omega_s$ \\
filtre passe-bas : on garde que le signal $\omega_r-\omega_s$
expérience

\subsection{Élargissement spectral}



\section*{Conclusion}

décalage longueur d'onde d'une source lumineuse vers le rouge quand elle s'éloigne \\
application Doppler en médecine : radio qui mesure la vitesse du sang dans vaisseaux sanguins, ultrasons \\


\begin{remarques} \begin{itemize} 
\item obligé de le mettre en niveau L3 et de faire relativiste
\item l'expérience est incontournable
\item ODG
\item important : notion d'invariance, on change de référentiel, on retrouve la même physique, principe de relativité
\item vitesse relative permet de ne traiter qu'un sens
\item physiquement, sont-ce deux cas différents ? non, l'un ou l'autre est immobile, le référentiel est galiléen ou non, sans importance, c'est la vitesse d'entraînement de l'un par rapport à l'autre
\item référentiel : ensemble de coordonnées d'espace-temps qui ont le même temps
\item être clair sur la définition des vitesses, référentiels, hypothèses
\item onde de choc en hydrodynamique : discontinuité de la pression qui se propage, voir la théorie
\item décalage vers le rouge : contexte astrophysique non cosmologique
\item dilatation des temps : démo avec un photon qui est réfléchi et qu'on "récupère" avec une vitesse de l'émetteur/récepteur
\item pendant une période, source bouge
\item faire modèle Newtonien de l'univers en expansion ?
\item référentiel d'inertie : seule configuration où omega et k sont différents est l'effet Doppler 
\item référentiel accéléré : le nombre de particules détectées (étude quantique) dépend du référentiel, à la base du rayonnement thermique des trous noir
\item effet Doppler sur ondes de matière : ondes sismiques, voir application
\item radar
\item Doppler application en médecine
\end{itemize} \end{remarques}
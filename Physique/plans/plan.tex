\documentclass[12pt,french,a4paper,oneside]{scrbook}
\usepackage[top=2cm, bottom=3cm, left=1cm, right=3cm, heightrounded,
  marginparwidth=2cm, marginparsep=3mm]{geometry}
\usepackage{fontspec}
\setmainfont{Linux Libertine O}
\usepackage[french]{babel}
\usepackage[bookmarks=true]{hyperref}
\usepackage{amsmath}
\usepackage{amssymb}
\usepackage{amsthm}
%\usepackage{commath}
\usepackage[dvipsnames]{xcolor}
\usepackage[framemethod=tikz]{mdframed}
\usepackage{sectsty}
\usepackage{csquotes}
\usepackage{wrapfig}
\usepackage{ragged2e}
\usepackage[locale=FR, exponent-product=., inter-unit-product =\cdot, binary-units]{siunitx}
\usepackage{etex}
\usepackage{m-pictex,m-ch-en}
\usepackage{mhchem}
\usepackage{fourier}
\usepackage{interval}
\usepackage{graphicx}
\usepackage{circuitikz}
\usepackage{esvect}
\usepackage{mathtools}
\usepackage{sectsty}
\usepackage{physics}
\usepackage{url}
\usepackage[most]{tcolorbox}
\usepackage{tikz}
\usetikzlibrary{calc,patterns,decorations.pathmorphing,decorations.markings}

\usepackage[backend=biber,
            bibencoding=inputenc,
            hyperref=auto,
            refsection=chapter]{biblatex}
\addbibresource{biblio.bib}

\setcounter{tocdepth}{0}

% Définition du style

\hypersetup{
	colorlinks   = true, %Colours links instead of ugly boxes
	linkcolor={red!50!black},
	citecolor={blue!50!black},
	urlcolor={blue!80!black}
}

\definecolor{mybluei}{RGB}{28,138,207} % Cadre leçon
\definecolor{myblueii}{RGB}{131,197,231} % Titre leçon
\definecolor{myblueiii}{RGB}{11,53,79} % Bleu foncé cadre Niv, Message, Biblio, Prérequis
\definecolor{myblueii}{RGB}{19,61,87} % Titre leçon
\definecolor{mybluei}{RGB}{27,69,95} % Cadre leçon

\definecolor{myorange}{RGB}{255,95,0}
\definecolor{myviolet}{RGB}{79,6,156}
\definecolor{mygreen}{RGB}{48,130,11}
\definecolor{myred}{RGB}{215,51,51}
\definecolor{myyellow}{RGB}{207,162,41}
\definecolor{mysalmon}{RGB}{215,101,51}

\mdfdefinestyle{bibliostyle}{frametitle={\tikz[baseline=(current bounding box.east),outer sep=0pt]\node[anchor=east,rectangle,fill=myblueiii]{\color{white}Bibliographie};},innertopmargin=10pt,linecolor=myblueiii,linewidth=2pt,topline=true,frametitleaboveskip=-10pt,}

\mdfdefinestyle{remstyle}{frametitle={\tikz[baseline=(current bounding box.east),outer sep=0pt]\node[anchor=east,rectangle,fill=mybluei]{\color{white}Remarques};},innertopmargin=10pt,linecolor=mybluei,linewidth=2pt,topline=true,frametitleaboveskip=-10pt,}

\mdfdefinestyle{alertstyle}{frametitle={\tikz[baseline=(current bounding box.east),outer sep=0pt]\node[anchor=east,rectangle,fill=myred]{\color{white}Attention};},innertopmargin=10pt,linecolor=myred,linewidth=2pt,topline=true,frametitleaboveskip=-10pt,}

\addtokomafont{disposition}{\usefont{T1}{qhv}{b}{n}\selectfont}

\addtokomafont{chapter}{\fontsize{22pt}{22pt}\color{myblueii}}
\newkomafont{chapternumber}{\fontsize{50}{120}\selectfont\color{white}}
\newkomafont{chaptername}{\itshape\rmfamily\small\color{white}}
%\addtokomafont{chapterentry}{\normalcolor}% entrys in tableofcontents not blue

\addtokomafont{section}{\fontsize{14pt}{14pt}\selectfont}
\newkomafont{sectionnumber}{\fontsize{18pt}{18pt}\selectfont\rmfamily\color{white}}

\addtokomafont{subsection}{\fontsize{12pt}{12pt}\selectfont}
\newkomafont{subsectionnumber}{\fontsize{16pt}{16pt}\selectfont\rmfamily\color{white}}

\setcounter{secnumdepth}{\subsectionnumdepth}% subsubsection and lower unnumbered

% \renewcommand\chapterformat{%
%   \raisebox{-6pt}{\colorbox{mybluei}{%
%       \parbox[b][60pt]{45pt}{\centering%
%         {\renewcommand{\chaptername}{Montage}\usekomafont{chaptername}{\chaptername}}%
%         \vfill{\usekomafont{chapternumber}{\thechapter\autodot}}%
%         \vspace{6pt}%
%       }}}\enskip}

\newcommand*{\chapmark}{%
  \renewcommand{\chaptername}{Leçon}
  \begin{tcolorbox}[title=\centering\usekomafont{chaptername}{\chaptername},colframe=mybluei,colback=mybluei,hbox,sharp corners,left=1mm, right=1mm]
    \centering\usekomafont{chapternumber}{\thechapter\autodot}
  \end{tcolorbox}%
}
\renewcommand*{\chapterformat}{%
  \begin{minipage}[c]{\widthof{\chapmark}}
    \chapmark
  \end{minipage}%
}

\NewDocumentCommand\Chapter{o m}{% note the uppercase "C"
  \IfValueTF{#1}% optional argument given or not
  {% with optional argument:
    \chapter[#1]{%
      \begin{minipage}[b]{\textwidth-\widthof{\chapmark}}

      #2
      \end{minipage}}%
  }{% without optional argument:
    \chapter[#2]{%
      \begin{minipage}[b]{\textwidth-\widthof{\chapmark}}
      #2
      \end{minipage}}%
  }
}

\chapterfont{\color{myblueii}}
\partfont{\color{myorange}}

\renewcommand\thesection{\arabic{section}}

% Définition des macros

\newcommand{\iu}{\mathrm{i}\mkern1mu}
\newcommand{\ju}{\mathrm{j}\mkern1mu}
\renewcommand\div[1]{\mathrm{div}\,#1}
\newcommand{\dive}{\mathrm{div}\,}
\newcommand{\ddroit}{\mathrm{d}\,}

\renewcommand\grad[1]{\vv{\mathrm{grad}}\,#1}
\newcommand\rot[1]{\vv{\mathrm{rot}}\,#1}
%\DeclareMathOperator\erf{erf}
%\DeclareMathOperator{\sign}{sign}

%\newcommand{\Arch}{\operatorname{\mathit{A\kern-.06em r}}} % http://de.wikipedia.org/wiki/Archimedes-Zahl
%\newcommand{\Biot}{\operatorname{\mathit{B\kern-.06em i}}} % http://de.wikipedia.org/wiki/Biot-Zahl
%\newcommand{\Cauc}{\operatorname{\mathit{C\kern-.07em a}}} % http://de.wikipedia.org/wiki/Cauchy-Zahl
%\newcommand{\Damk}{\operatorname{\mathit{D\kern-.06em a}}} % http://de.wikipedia.org/wiki/Damk%C3%B6hler-Zahl
%\newcommand{\Eule}{\operatorname{\mathit{E\kern-.03em u}}} % http://de.wikipedia.org/wiki/Euler-Zahl
%\newcommand{\Four}{\operatorname{\mathit{F\kern-.10em o}}} % http://de.wikipedia.org/wiki/Fourier-Zahl
%\newcommand{\Frou}{\operatorname{\mathit{F\kern-.07em r}}} % http://de.wikipedia.org/wiki/Froude-Zahl
%\newcommand{\Gras}{\operatorname{\mathit{G\kern-.05em r}}} % http://de.wikipedia.org/wiki/Grashof-Zahl
%\newcommand{\Karl}{\operatorname{\mathit{K\kern-.11em a}}} % http://de.wikipedia.org/wiki/Karlovitz-Zahl
%\newcommand{\Knud}{\operatorname{\mathit{K\kern-.11em n}}} % http://de.wikipedia.org/wiki/Knudsen-Zahl
%\newcommand{\Lewi}{\operatorname{\mathit{L\kern-.05em e}}} % http://de.wikipedia.org/wiki/Lewis-Zahl
%\newcommand{\Mach}{\operatorname{\mathit{M\kern-.10em a}}} % http://de.wikipedia.org/wiki/Mach-Zahl
%\newcommand{\Nuss}{\operatorname{\mathit{N\kern-.09em u}}} % http://de.wikipedia.org/wiki/Nusselt-Zahl
%\newcommand{\Pecl}{\operatorname{\mathit{P\kern-.08em e}}} % http://de.wikipedia.org/wiki/P%C3%A9clet-Zahl
%\newcommand{\Pran}{\operatorname{\mathit{P\kern-.03em r}}} % http://de.wikipedia.org/wiki/Prandtl-Zahl
%\newcommand{\Rayl}{\operatorname{\mathit{R\kern-.04em a}}} % http://de.wikipedia.org/wiki/Rayleigh-Zahl
%\newcommand{\Reyn}{\operatorname{\mathit{R\kern-.04em e}}} % http://de.wikipedia.org/wiki/Reynolds-Zahl
%\newcommand{\Schm}{\operatorname{\mathit{S\kern-.07em c}}} % http://de.wikipedia.org/wiki/Schmidt-Zahl
%\newcommand{\Sher}{\operatorname{\mathit{S\kern-.07em h}}} % http://de.wikipedia.org/wiki/Sherwood-Zahl
%\newcommand{\Stro}{\operatorname{\mathit{S\kern-.07em r}}} % http://de.wikipedia.org/wiki/Strouhal-Zahl
%\newcommand{\Webe}{\operatorname{\mathit{W\kern-.14em e}}} % http://de.wikipedia.org/wiki/Weber-Zahl

%\newcommand{\biblio}{\begin{mdframed}[style=bibliostyle]\relax \printbibliography[heading=none]\end{mdframed}}
\newcommand{\biblio}{\hspace*{-\parindent}\colorbox{myblueiii}{\textcolor{white}{Bibliographie}}
\printbibliography[heading=none]
}

\newcommand\lemessage[1]{
  \hspace*{-\parindent}\colorbox{myblueiii}{\textcolor{white}{Message}}\hspace{5pt} #1 \vspace{0.5em}
}

\newenvironment{remarques}{
  \begin{mdframed}[style=remstyle]
}
{\end{mdframed}}

\newenvironment{attention}{
  \begin{mdframed}[style=alertstyle]
}
{\end{mdframed}}

\newcommand\nivpre[2]{
\hspace*{-\parindent}%
\begin{minipage}[t]{0.5\textwidth}
  \colorbox{myblueiii}{\textcolor{white}{Niveau}}\hspace{5pt} #1
\end{minipage}
\begin{minipage}[t]{0.5\textwidth}
  \colorbox{myblueiii}{\textcolor{white}{Prérequis}}\hspace{5pt} #2
\end{minipage}
\vspace{0.5em}
}

\newcommand\transition[1]{\hspace*{-\parindent}\colorbox{mygreen}{\textcolor{white}{Transition :}}\hspace{5pt} #1}

\newenvironment{ecran}{
   \begin{mdframed}[%
    frametitle={\color{myviolet}Écran},
    frametitlealignment={},
    frametitlerule=false,
    topline=false,
    bottomline=false,
    nobreak=true,
    linewidth=1mm,
    leftline=true,
    linecolor=myviolet,
    innerleftmargin=10pt,
]
}{%
\end{mdframed}
}

\newenvironment{experience}{
   \begin{mdframed}[%
    frametitle={\color{mysalmon}Expérience},
    frametitlealignment={},
    frametitlerule=false,
    topline=false,
    bottomline=false,
    nobreak=true,
    linewidth=1mm,
    leftline=true,
    linecolor=mysalmon,
    innerleftmargin=10pt,
]
}{%
\end{mdframed}
}

\makeatletter
\def\blx@refpatch@chapter#1{%
	\ifundef\chapter
	{\blx@err@nodocdiv{chapter}}
	{\pretocmd\@makechapterhead{#1}
		{}
		{\blx@err@patch{\string\@makechapterhead}}
		\pretocmd\@makeschapterhead{#1}
		{}
		{\blx@err@patch{\string\@makeschapterhead}}}}
\makeatother

% change allure de la table des matières : 
\usepackage{tocloft}
\renewcommand{\cftchapfont}{\scshape}
\renewcommand{\cftsecfont}{\bfseries}
\renewcommand{\cftfigfont}{Figure }
\renewcommand{\cfttabfont}{Table }
%

\title{Leçons de physique}
\author{Ludivine Emeric}  % Mettre l'auteur ici
\date{\today}  % Mettre la date si besoin

\begin{document}
\maketitle

{
	\hypersetup{
		linkcolor={black},
	}
	\tableofcontents
}

%\part{Transverses}
\Chapter{Symétries}

\nivpre{Licence}{
 \begin{itemize}
  \item électrostatique
  \item mécanique lagrangienne
  \item quadri-forces
  \item théorie de la bifurcation (pour diagrammes d'Elastica)
 \end{itemize}
}

\lemessage{Ici y a trop de choses, effet catalogue, ça part trop loin.}

\biblio{}

\section*{Introduction}
de tout temps physiciens se sont intéressés à symétrie \\
Euclide...

\section{Principe de symétrie}
\subsection{Qu'est-ce qu'on entend par symétrie}
symétrie planaire \\
Chimie : molécules chirales, prendre les formes à assembler \\
d'autres types de symétrie : symétries continues, associées à notions d'invariance \\
invariance par translation dans l'espace : ex lacher balle de ping-pong donne même résultat à un endroit ou à un autre \\
invariance par translation dans le temps : même expérience dans le temps \\

invariance par rotation \\

\subsection{Principe de Curie}
(Pierre Curie) \\
les effets d'un phénomène possèdent au moins les symétries de sa cause \\
électrostatique : cause distribution de charges $\rho$, effet : champ électrique $\vec{E}$ \\
symétries de $\rho$ d'une sphère chargée : ne dep pas du temps, inv par rotation $\rightarrow$ même chose pour $\vec{E}$ \\
résolution avec Gauss... \\
Remarque principe de Curie en chimie : mélange racémique, pour produire un seul énantiomère il faut briser la symétrie, introduire autre chose ; importance par exemple avec thalidomine tératogène vs anti-nauséeux, corps humain est lui même chiral \\


\subsection{Brisure de symétrie}
Elastica : flambage, position d'équilibre penchée (bifurcation) \\
Généralisation : il faut prendre en compte l'ensemble des effets possibles \\
diagramme de bifurcation : diagramme $x_{eq}$ en fonction de m \\ 
limitations ici : l'elastica est un peu déformé, tend vers un côté préférentiellement \\

\section{Symétries et lois de conservation}


\subsection{Conservation de l'impulsion}
$F=-\frac{\ddroit V}{\ddroit x}$ F dérive du potentiel V \\
particule de masse m dans un référentiel galiléen \\
PFD \\
hypothèse : le système est invariant par translation dans l'espace  \\
$\frac{\ddroit}{\ddroit x} \rightarrow 0$ \\
F=0 \\
$\frac{\ddroit(m\dot{x}}{\ddroit t} =0$ \\
l'impulsion se conserve \\

\subsection{Théorème de Noether}
lagrangien $L(x,\dot{x},t)$ avec x(t) et $\dot{x} (t)$ \\
hypothèse : le système est invariant par variation continue d'une grandeur S : $\frac{\ddroit L}{\ddroit S}=0$ \\

\subsection{Conservation de l'énergie}
invariance par translation dans le temps : $s=t$ \\
Hamiltonien...\\
calcul montre que $E_m$ se conserve \\



\section{Autres invariances}
\subsection{Invariance par rapport au choix de coordonnées}
la nature se fiche de notre choix de coordonnées \\
mécanique classique : force est un vecteur, $m\vec{a}$ \\
relativité restreinte : quadri-vecteurs \\

\subsection{Invariance du choix des unités}
 théorème $\Pi$ \\
 analyse dimensionnelle \\

dilatation des systèmes \\


\section*{Conclusion}
on peut en déduire bcp d'informations sur les propriétés que doit avoir un système \\
contraintes sur modèles théoriques \\
théorie quantique des champs \\


\begin{remarques} \begin{itemize} 
\item invariances et symétries : invariance est conséquence de symétrie
\item Curie a découvert le principe par cristallographie : cristaux ont propriétés de symétrie discrète ; effets : interaction avec champ électromagnétique, diffraction aux rayons X ; voir BUP 689
\item inversion du temps : important dans beaucoup de domaines, réversibilité
\item invariance par rotation vs symétrie par rotation
\item la chimie ne se résume pas aux effets chiraux, autres phénomènes en chimie qui jouent sur les réactions : effets stériques, orbitaux
\item manipulation sur coin de table : important pour montrer que c'est concret, analogies ; genre pendule dépend du temps mais invariant dans le temps
\item thermodynamique : transition de phase ferromagnétisme paramagnétisme $\rightarrow$ brisure de symétrie, bifurcations fourche
\item lien avec entropie : augmente lors de la brisure
\item x : coordonnées généralisée, par forcément une longueur
\item s : paramètre qui décrit la symétrie, nom ?
\item Hamiltonien est énergie du système toujours ? quand dimension d'une énergie oui
\item contexte historique Einstein : équation de Maxwell étaient invariant de Lorentz, pas Galilée, en relativité l'invariant est l'intervalle 
\item théorème $\Pi$ : si j'ai une fonction de variables dimensionnées, je peux le réécrire come une relation adimensionnée
\item symétrie de jauge
\item invariant de l'électromagnétisme et Noether : conservation de la charge
\item Elastica : imperfect pitchfork
\item faire calcul d'entropie ?
\item principe de relativité affirme que les lois physiques s'expriment de manière identique dans tous les référentiels inertiels : les lois sont « invariantes par changement de référentiel inertiel »
\end{itemize} \end{remarques}


\Chapter{Adaptation d'impédance}

\nivpre{CPGE}{
 \begin{itemize}
  \item équation de d'Alembert
   \item modélisation du câble coaxial
   \item AO
 \end{itemize}
}

\lemessage{Partout, c'est la vie.}

\biblio{}

\section*{Introduction}

Permettre un transfert de puissance optimal \\
Couche anti-reflet, cornets, cuivre (instrument), picots en salle insonorisée ou au plafond de salles... \\

\marginpar{réf. \cite{Berkeley_ondes}, \cite{Thibierge_ondes}}

Voir Berkeley, Landau et Feynman \\

\section{Câble coaxial : réflexion sur une impédance terminale}
\subsection{Rappel}
équations des télégraphes sans pertes \\
\subsection{Réflexion sur une résistance terminale}
Résistance placée à la suite du câble \\
Détermination des coefficients de réflexion en amplitude \\
$R=Z_C$ : réflexion nulle, c'est ce qu'on appelle adaptation d'impédance \\
intérêt du montage suiveur \\
\section{Application pratique en acoustique : l'échographie}
\subsection{Interface air-muscle}
intensité réfléchie, intensité transmise \\
$R=\left( \frac{Z_2 - Z_1}{Z_2 + Z_1} \right)^2 $ \\
$T=4 \frac{Z_2 Z_1}{\left( Z_2 + Z_1\right)^2} $  \\
$Z_\mathrm{air}=444 kg.m^2.s^{-1}$ et $Z_\mathrm{muscle}=1,7 . 10^6 kg.m^2.s^{-1}$ \\
$T=10^{-3}$ très faible

\subsection{Gel entre les deux}
a priori, impédance du gel est entre les 2 \\
Attention, le calcul est bizarre : ne pas se fier au td d'hydro \\
en supposant qu'il n 'y a pas de réflexion :
condition d'adaptation d'impédance $\rightarrow$ $\frac{Z_g-Z_a}{Z_g+Z_a}=\frac{Z_g-Z_m}{Z_g-Z_m} e^{-2 i k_g e}$ \\
les impédances ne sont pas complexes car équation vraie que pour certains e \\

conditions aux limites $\rightarrow$  $2 k_g e=2n\pi$ ou $(2n+1) \pi$, paire ou impaire $\rightarrow$ solution intéressante : impaire \\
$Z_g=\sqrt{Z_a Z_m}$ \\
$Z_g=2.7 .10^4 kg.m^{-2}.s^{-1}$

\section*{Conclusion}
maximiser intensité transmise \\

\begin{remarques} \begin{itemize} 
\item exo de l'échographie est schématique : en vrai ce n'est pas de l'air, c'est l'émetteur
\item pourquoi il n'y a pas de réflexion dans l'air ? 
\item Fabry-Pérot, couche anti-reflet
\item suiveur : désadapte l'impédance
\item électronique : on parle d'adaptation d'impédance pour maximiser le rendement
\item quand on cherche à annuler la réflexion, on maximise la puissance transmise
\item impédance d'un dipôle : extensif
\item impédance d'un milieu : intensif 
\item on les traite différemment : ça dépend de la dimension à laquelle on les regarde
\item impédance caractéristique
\item cornet : (ex ondes centimétriques) fait varier continûment l'impédance
\item impédance : lien entre une force cinétique et une force de rappel ; oscillation de l'énergie par passage de forme cinétique à potentielle 
\item coefficients de Fresnel
\item $Z=\frac{"force"}{"déplacement"}$
\item conditions aux limites : dépend du rapport entre les impédances, on peut donc tout ramener aux impédances, ça rebondit
\item cas des dipôles : traité comme ponctuel
\item trompette, réflexion du son sur un mur ; boîte d'oeuf : tout est absorbé
\item déferlement de vagues ?
\item 
\end{itemize} \end{remarques}
\Chapter{Conversions d'énergie}

\nivpre{CPGE}{
 \begin{itemize}
  \item premier et deuxième principes de la thermodynamique
   \item lois de Faraday et de Lenz
   \item action d'un champ magnétique sur un moment magnétique
 \end{itemize}
}

\lemessage{Il n'y a pas de source d'énergie brute dans la nature (genre l'hélium c'est génial youhou, en fait non), il n'y a que des conversions et transports, et puis des pertes... beaucoup de pertes... partout des pertes !}

\biblio{}

\section*{Introduction}
L'énergie est une grandeur définie comme "se conservant" \\
différentes formes, on en veut sous différentes formes, comment passer d'un à l'autre ? \\ 
Comment peut-on l'utiliser ?  \\
Sa consommation permet de faire fonctionner tous les outils de la vie \\
électrique, thermique, mécanique, chimique... \\
 on va s'intéresser à la chaîne majoritaire de production d'électricité en France : combustion $\rightarrow$ énergie thermique $\rightarrow$ ???  $\rightarrow$ énergie mécanique $\rightarrow$  ???  $\rightarrow$ énergie électrique \\
 on va voir ce que sont ces ???
\marginpar{réf. \cite{Neveu_moteurs}}

\section{Conversion de l'énergie thermique en énergie électrique}
\subsection{Rappels de thermodynamique}
1er principe, variation d'énergie interne : travail et transfert thermique \\
moteur thermique permet de passer d'une forme à l'autre \\
2e principe, variation d'entropie : entropie échangée et entropie créée \\

\subsection{Moteurs thermiques}
système qui transforme Q en W \\
exemple du moteur de Stirling \\
source chaude, source froide \\
Transformations cycliques : $\Delta S=0$, $\Delta U=0$ \\

\subsection{Rendement}
$Q_f<0$ \\
$\eta = \frac{-W}{Q_c}$  \\
%$\eta <= 1-\frac{T_f}{T_c}$ : égalité pour rendement de Carnot, $41%$ en théorie ici mais plutôt $0.5%$ en pratique \\

\subsection{Machines thermiques réelles}
Pas réversibles : il peut y avoir des frottements et échauffement de la source froide \\
Moteur à explosion \\
Moteur diesel : animation avec admission, compression, combustion, échappement \\
\href{https://youtu.be/CyIz7AfiF04}{Vidéo Youtube moteur} \\
\href{https://youtu.be/DKF5dKo_r_Y}{ou cette vidéo} \\
\href{https://www.youtube.com/watch?v=aqfzJDOQl7M}{ou celle-ci} \\
transition : chaîne énergétique, on a vu comment faire thermique vers mécanique, on a de l'énergie mécanique là, on veut de l'énergie électrique

\section{Conversion de l'énergie mécanique en énergie électrique}
\subsection{Principe d'un alternateur}
expérience avec une bobine Leybold, une barre de fer doux, un aimant tourne au milieu \\ 
quand ça tourne, on a création d'un champ électrique \\
flux du champ magnétique, fem \\
loi de Lenz : champ $\vec{B}'(t)$ créé s'oppose à la rotation des aimants \\
un couple résistif apparaît : $\vec{\Gamma}=vec{M} \wedge vec{B'}$ \\

\subsection{Moteur synchrone}
$\vec{\Gamma}=M B \sin \theta \vec{u_z}$ \\
B : champ tournant à $\Omega$ \\
M : rotor à $\omega$ \\
initialement $\theta(0)=\alpha$ \\
$\theta (t)=\left( \Omega -\omega \right) t + \alpha $ \\
$\vec{\Gamma} (t)= M B \sin \left( (\Omega-\omega)t+\alpha \right)$ \\
moyenne nulle si les fréquences sont différentes \\
si $\Omega=\omega$, non nul... courbe caractéristique de $\Gamma=f(\alpha)$  \\
cas alternateur \\
cas moteur : on regarde la branche moteur ($\Gamma >0$), on identifie laquelle est pente est stable ou instable \\
pente positive : stable, pente négative : instable
car si $\Gamma_r$ augmente, alors $\omega$ diminue, donc $\alpha$ augmente

\section*{Conclusion}
Chaîne énergétique totale \\
TD : sources de pertes au sein d'un alternateur \\
énergie géothermique, solaire \\


\begin{remarques} \begin{itemize} 
\item fonction d'état : définie à l'équilibre, dépend des variables d'état 
\item on peut parler de variables d'état primitives (V,N,U) (une fonction d'état ne dépend pas que de ça) mais pas en CPGE
\item définition énergie interne : ensemble des énergies cinétiques et potentielles microscopiques, définies par statistique
\item échange de matière entre systèmes, que devient l'entropie ? 
\item source chaude : modèle du thermostat
\item réversibilité n'est pas souhaitable car puissance tend vers 0
\item moteur diesel : pourquoi plusieurs cylindres ? permet d'améliorer régularité, éviter les acoups
\item on peut faire rentrer le combustible dans le phase d'admission
\item alternateur 
\item vraie centrale : on chauffe de l'eau, s'évapore, pression sur turbines, met en rotation l'arbre de l'alternateur (rotor)
\item préciser Stirling : 2 monothermes + 2 adiabatiques
\item autre plan possible : fusion dans soleil, effet photovoltaïque
\end{itemize} \end{remarques}
\include{LPn40}
\include{LPn16}
\include{LPn30}
\Chapter{Phénomènes non linéaires}

\nivpre{CPGE}{
 \begin{itemize}
  \item 
    \item 
      \item 
    \item 
      \item 
    \item 
 \end{itemize}
}

\lemessage{.}

\biblio{}

\section*{Introduction}

%\marginpar{réf. \cite{ToutenunPC_2016},\cite{hecht},\cite{huard_pola}}

\section{}
\subsection{}
\subsection{}
\section{}
\subsection{}
\subsection{}
\section{}
\subsection{}
\subsection{}



\section*{Conclusion}


\begin{remarques} \begin{itemize} 
\item 
\item 
\item 
\item 
\item 
\item 
\end{itemize} \end{remarques}



%\part{Mécanique}
\Chapter{Gravitation}

\nivpre{CPGE}{
 \begin{itemize}
  \item Mécanique de point
  \item Théorèmes généraux mécanique
  \item Électrostatique
 \end{itemize}
}

\lemessage{Attention demo trajectoires coniques n'est plus au programme de CPGE depuis 2014, nécessité de placer en L3 ou admettre le résultat sans trop analyser mathématiquement, mais les trajectoires sont à connaître sans les formules géométriques.}

\biblio{}


\section*{Introduction}
modèle géocentrique, planètes ont été identifiées : se déplacent dans le ciel nocturne \\
modèle héliocentrique : Copernic \\
\href{https://solarsystem.nasa.gov/solar-system/our-solar-system/overview/}{animation NASA}
ou
\href{http://www.planete-astronomie.com/animation-de-la-position-des-planetes.html}{animation astronomie}




\section{Interaction gravitationnelle}
\subsection{Force gravitationnelle}
mentionner troisième loi de Newton : actions réciproques
\subsection{Champ de pesanteur}
\subsection{Énergie potentielle gravitationnelle}
\section{Mouvement dans un potentiel gravitationnel}
\subsection{Position du problème}
\subsection{Potentiel effectif}
démo coniques seulement si choix L3
\subsection{Lois de Kepler}
\section{Application : vitesse de libération}
\section*{Conclusion}

%\begin{ecran}
%	Contenu affiché sur diapositives
%\end{ecran}
%
%\begin{remarques}
%	Remarque concernant le contenu
%\end{remarques}
%
%\transition{Belle transition entre deux parties ou sous-parties.}
%
%\begin{itemize}
%	\item Calcul, définition, etc. : le contenu de la leçon 
%\marginpar{réf. \cite{ToutenunPCSI_2003}}
%	\item \begin{equation*}
%		\rot{E} = - \pdv{\vv{B}}{t} \qq{et} \oiint_\mathcal{S} \vv{E} \cdot \vv{\dd{S}} = \frac{Q_\text{int}}{\varepsilon_0} \,.
%	\end{equation*}
%\end{itemize}
%
%\begin{experience}
%	Expérience pour illustrer
%\end{experience}
%
%\begin{attention}
%	Erreur faite souvent ou point sur lequel insister.
%\end{attention}


\begin{remarques} \begin{itemize}
\item unité de G ? mesure de G ? pendule de torsion, horloge atomique
\item masses ponctuelles ?
\item traditionnellement ce que l'on appelle pesanteur sur Terre : on inclut force (centrifuge) d'inertie d'entrainement (référentiel terrestre non galiléen, Terre tourne sur elle-même) et force de marée 
\item analogue pour champ magnétique B dans gravitation ? non
\item modèle de la Terre creuse : roman de Jules Verne... alors par de gravitation à l'intérieur (Gauss, masse nulle entourée donc force nulle)
\item rotG ? nul
\item pas de masse négative : force gravitationnelle d'un anti-atome d'hydrogène, interagissent de la même manière par interaction gravitationnelle ? en cours de recherche
\item on peut négliger variations gravitation à la surface de la Terre, mesure ? gravimétrie, applications économiques : recherche pétrole, mesure différence de densité
\item mesure de la chute libre par Galilée : tour de Pise, pas de chronomètre, mesure du temps à l'aide de son pouls
\item détecter une planète : voir changement de trajectoires, clignotement quand passent devant leur étoile, lunette gravitationnelle, méthode des transits 
\end{itemize} \end{remarques}
\include{LP02}
\include{M1}
\include{M2}
\Chapter{Précession et approximation gyroscopique}

\nivpre{CPGE}{
 \begin{itemize}
  \item A
 \end{itemize}
}

\lemessage{Message important à faire passer lors de la leçon.}

\biblio{}

\section*{Introduction}

\begin{ecran}
	Contenu affiché sur diapositives
\end{ecran}



\transition{Belle transition entre deux parties ou sous-parties.}



\begin{experience}
	Expérience pour illustrer
\end{experience}

\begin{attention}
	Erreur faite souvent ou point sur lequel insister.
\end{attention}


\begin{remarques}
 \begin{itemize}
  \item toupie
  \item frisbee, boomerang : trainée et portance permettent le vol, effet gyroscopique le stabilise
 \end{itemize}	
\end{remarques}
\include{M4}
\include{M5}
\Chapter{Collisions et lois de conservation en mécanique classique et relativiste}

\nivpre{L3}{
 \begin{itemize}
  \item mécanique du point
  \item changement de référentiel
  \item relativité, quadrivecteurs, loi de la dynamique
 \end{itemize}
}

\lemessage{Socle de l'étude de la dynamique des réactions nucléaires et des collisions de particules (désintégration, collision, etc.).}

\biblio{}

\section*{Introduction}
il y a partout des collisions \\
on reste sur le cas élastique (conservation énergie cinétique, pas transfert énergie interne, cf boules de billard) inélastique=déformation en général (cf 2 voitures) \\
montrer pendule de Newton \\
plein d'images \\
\href{https://www.youtube.com/watch?v=4v2RHtBTbj8}{Vidéo Youtube du choc de 2 billes} \\
faire que des points matériels ou sphères dures \\
A quoi servent les collisions ? sonder la matière \\

\section{Conservation de l'énergie}
exp de rutherford : particule alpha sur cible particule Au, totalement classique (une particule ponctuelle, l'autre non et on détermine rayon de l'atome d'Au) \\
on sonde la matière qu'on fait collisionner \\

\section{Conservation du quadrivecteur énergie-impulsion}
diffusion compton : photon sur électron, diffusion \\


\bigskip
Seulement si le temps : 
\section{Collision inélastique : boson de Higgs}
découverte du boson de Higgs \\
2 gluons fusionnent, se désintègre en 2 photon (symétriques), 2 hbar k/c, à partir du pic d'énergie en fct de masse \\
trouver figure sur google

\section*{Conclusion}


% conservation de l'énergie
% 		exp de rutherford : alpha sur particule en Au, totalement classique (une particule ponctuelle, l'autre non et on détermine rayon de l'atome d'Au), on sonde la matière qu'on fait collisionner
%					modèle sphères dures ? solides non déformables ok
% conservation du quadrivecteur énergie-impulsion 
% 		diffusion compton : photon sur électron, diffusion d'un photon
% collision inélastique : découvrte du boson de Higgs
			% 2 gluons fusionnent, se désintègre en 2 photon (symétriques), 2hbark/c, à partir du pic d'énergie en fct de masse

\begin{remarques} \begin{itemize} 
\item en pratique, pour analyser une réaction, on fait un usage intense du fait que, d’une part, le quadrivecteur énergie-impulsion est conservé, et que, d’autre part, les pseudo-produits scalaires de quadrivecteurs sont invariants par changement de référentiels
\item référentiel du centre de masse : impulsion totale est nulle (mais pas l'énergie)
\item la masse invariante M est un invariant de Lorentz : pour un système donné, sa valeur est indépendante du référentiel.
\item certaines collisions inélastiques en classique sont élastiques en relativiste
\item symétrie $\Rightarrow$ conservation d'une quantité
\item ne pas faire trop de calculs
\item ne pas dissocier classique et relativiste, traiter ensemble
\end{itemize} \end{remarques}
\Chapter{Manifestations du caractère non galiléen d'un référentiel}

\nivpre{Licence}{
 \begin{itemize}
  \item mécanique du point (cinématique)
  \item force d'attraction newtonienne
  \item changement de référentiel, point coïncident (=point fictif que l'on balade entre les référentiels)
  \item principe d'inertie
 \end{itemize}
}

\lemessage{Attention le référentiel terrestre n'est plus vraiment étudié en CPGE et changement de référentiel est en spé. Ne pas faire trop de théorie, juste rappel sur slide.}

\biblio{}

\section*{Introduction}

\section{Description des référentiels usuels}
\subsection{Principe d'inertie}
\subsection{Changement de référentiel et forces d'inertie}
Rappels
\subsection{Une définition relative}

\section{Application au référentiel terrestre}
\subsection{Force d'inertie d'entraînement}
\subsection{Force d'inertie de Coriolis}
\section*{Conclusion}



\begin{remarques} \begin{itemize} 
\item trop de rappels, balancer plutôt les lois de composition et faire un schéma pour le point H dans $-\Omega^2\vec{HM}$
\item autres exemples manifestations terrestre non galiléen : face de la lune est toujours la même, dérive des iceberg, Pluton et un de ses satellites sont toujours la même face l'une de l'autre, galaxies ont forme de spirale, Terre est élargie aux équateurs (implique $\Delta g=0,06$), déviation vers l'est, marées, cyclones
\item autres non galiléens : principe de l'essoreuse à salade, nucléaire pour séparer Rd... de Rd... ?
\item référentiel inertiel = référentiel galiléen
\item principe d'équivalence d'Einstein : masse inertielle et masse gravitationnelle sont les mêmes 
\item principe de relativité affirme que les lois physiques s'expriment de manière identique dans tous les référentiels inertiels : les lois sont « invariantes par changement de référentiel inertiel »
\item intéressant de parler des lavabos car idée reçue, vs cyclones, comparer dimensions... 
\item 
\end{itemize} \end{remarques}
\include{LPn26}
\include{LPn46}


%\part{Mécanique des fluides}

\include{LP03}
\include{LP04}
\include{LP05}
\include{LPn25}
\Chapter{Bilans macroscopiques en mécanique des fluides}

\nivpre{CPGE}{
 \begin{itemize}
  \item 
    \item 
      \item 
    \item 
      \item 
    \item 
 \end{itemize}
}

\lemessage{.}

\biblio{}

\section*{Introduction}

%\marginpar{réf. \cite{ToutenunPC_2016},\cite{hecht},\cite{huard_pola}}

\section{}
\subsection{}
\subsection{}
\section{}
\subsection{}
\subsection{}
\section{}
\subsection{}
\subsection{}



\section*{Conclusion}


\begin{remarques} \begin{itemize} 
\item on aurait pu traiter la fusée
\item fun fact pompiers : pour tenir le tuyau, un pompier se met dos à celui qui le tient pour l'aider à contrer la force 
\item ne pas faire de thermodynamique
\item 
\item 
\item 
\end{itemize} \end{remarques}
\Chapter{Physique du vol : portance et traînée}

\nivpre{CPGE}{
 \begin{itemize}
  \item mécanique des fluides
 \end{itemize}
}

\lemessage{.}

\biblio{}

\section*{Introduction}
On a vu montgolfière en statique des fluides \\
avion : vol, trainée et portance, comment ça marche \\
diapo : photo d'un avion Mirage \\
(toute la leçon est en régime écoulement stationnaire) \\
%\marginpar{réf. \cite{ToutenunPC_2016},\cite{hecht},\cite{huard_pola}}

\section{Force de traînée}
\subsection{Origines de la traînée}
Trainée est colinéaire et dans le sens de l'écoulement \\
\subsubsection*{Traînée de frottement} 
nombre de Reynolds $\mathrm{Re} \approx \frac{LU}{\nu}$ \\
écoulement libre loin d'une aile d'avion \\
écoulement visqueux proche (couche limite) caractérisé par épaisseur $\delta \approx \sqrt{\nu \tau} \approx \frac{L}{\sqrt{\mathrm{Re}}}$  \\
temps caractéristique $\tau=\frac{L}{U}$\\

\subsubsection*{Traînée de pression (ou traînée de forme)}
rappeler Bernoulli \\
définition du coefficient de traînée $\vec{F}_{tr}=-\frac{1}{2} \mu S U^2 C_x \left( \mathrm{Re} \right) \frac{\vec{U}}{U}$ \\
diapo écoulement en fonction du nombre de Reynolds : profils d'écoulements autour d'un cylindre circulaire pour différents $\mathrm{Re}$, $C_x$ en fonction de $\mathrm{Re}$ \\
forces de pression sont plus grandes là où attaque l'écoulement sur le cylindrique, plus faible en arrière : la résultante est vers l'arrière \\

ce sont les origines de la traînée \\

\subsection{Formule de Stokes}
$\mathrm{Re} << 10$ où $C_x \approx \frac{24}{\mathrm{Re}}$ \\
Force de traînée : expression \\
cas d'une sphère : $\mathrm{Re}=\frac{\mu 2 R U}{\eta}$ \\
Force de trainée, formule de Stokes : $\vec{F}_{tr}=-6\pi \eta R \vec{v}_{solide/fluide}$  ($\frac{24}{4}=6$)\\
Rq : pour $\mathrm{Re}$ intermédiaire $10^3 - 10^5 $, $C_x$ est pratiquement constant, la fonction de traînée est quadratique en $U$ \\
diapo : ordre de grandeur coefficients de traînée de différents objets 3D (coeff frontaux) \\

\subsection{Chute brutale du $C_x$ }
Au dessus d'un certain nombre de Reynolds (critique $\mathrm{Re_c}$), on a une chute du coefficient de traînée, c'est ce que l'on cherche en sport, fendre l'écoulement \\
En dessous de cette valeur, la traînée est trop grande, la balle lancée va moins loin \\
en augmentant la rugosité de la balle, on fait chuter le $\mathrm{Re_c}$ \\
exemple : balle de golf, rainures balle de baseball \\


\section{Force de portance}
\subsection{Portance d'une aile d'avion}
il existe plusieurs formes d'ailes d'avion, symétrique, asymétrique \\
schéma : bord d'attaque, bord de fuite, corde, angle d'inclinaison \\
portance est verticale dues aux différences de vitesses d'écoulement et donc différence de pression \\
force totale : oblique car frottements \\
diapo : coeff de traînée et de portance en fonction de l'angle d'inclinaison (portance, traînée et polaire d'un profil NACA 4412) \\
angle de décrochage \\
définition coefficient de portance : $\vec{F}_{pr}=\frac{1}{2} \mu S C_z U^2 \vec{n}$ \\
\subsection{Décrochage}
diapo : écoulement très peu décollé vs décrochage \\
\href{https://www.youtube.com/watch?v=zwiMvBMVFMw}{vidéo youtube} illustrant décrochage \\
expérience avec soufflerie et balance, on change l'inclinaison de l'aile : la balance va afficher un poids négatif si portance, maximale en valeur absolue si portance maximale, diminue si décrochage ; angle mesuré : $30^o$, dépend de la forme de l'aile \\

\subsection{Finesse aérodynamique}
$Finesse=\frac{C_z}{C_x}$ \\
$Finesse=\tan \alpha = \frac{AG}{AB}$ (c'est l'angle que fait la force totale avec la verticale) \\


\section*{Conclusion}
on peut parler de traîner induite lorsque l'aile n'est pas infiniment longue \\
la surpression en dessous rejoint la dépression au dessus \\
on utilise des winglets pour initier les tourbillons (traînée de condensation) \\ 
hélicoptères : faible vitesse, portance pas due aux mêmes phénomènes \\
la portance se retrouve vers le haut car pales incurvées, création d'une surpression en bas, dépression en haut
il y a aussi une traînée, cherche à diminuée \\
vol stationnaire : pas de traînée de la carcasse de l'hélico, seulement les pales \\

\begin{remarques} \begin{itemize} 
\item les pâles d'une éolienne sont vrillées
\item les pâles d'un hélicoptère : change d'incidence en court de course pour s'adapter, palonnier ; pale s'incurvent vers le haut par force centrifuge
\item couche limite : conditions aux limites de non-glissement (donc accrochage)
\item couche limite décolle au-delà d'un certain nb de Reynolds, l'écoulement va devenir turbulent, ce modèle n'est alors plus valable
\item pourquoi $C_z$ est adimensionnée avec $\frac{1}{2} \mu S U^2$ , pression, Bernoulli, surpression $1/2 \mu U^2$, le coefficient $C_z$ traduit l'écart à pression Bernoulli ; Gustave Eiffel a fait des mesures de temps de chute, a travaillé sur des souffleries pour mesurer les coefficients de traînée (fun fact : il ne croyait pas au nombre de Reynolds ni au changement de référentiel, c'était un ingénieur)
\item compressibilité du gaz est importante ? faut-il que l'air soit compressible pour voler ? empêche pas écoulement incompressible, sous l'eau ça marche, même proche la vitesse du son c'est négligeable ; ce qui compte c'est la différence de pression
\item $C_x$ coefficient de traînée est similaire entre sphère est cylindrique, seules les asymptotes changent, un peu plus hautes
\item on peut pas créer la portance à bas Reynolds 
\item il faut mentionner Bernoulli dans cette leçon
\item structure de l'aile : bord de fuite pointu pas nécessaire (avantage :  provoquer turbulences, intéressant pour traînée faible), bord d'attaque n'est pas nécessairement rond 
\item mécanisme pression plus élevée en bas d'une aile qu'en haut : écartement des lignes de courant en dessous, rapprochement au dessus ; flux de quantité de mouvement vers le base
\item surface considérée en portance : surface projetée (physique) ou surface totale à plat (utilisée par professionnels de l'aviation) 
\item intérêt, limites de l'expérience : tube de Pitot peut montrer vitesse uniforme en sortie de soufflerie, jusqu'à un certain point, aile d'avion est de même taille que soufflerie donc pas homogène, et angle d'inclinaison fait que pas uniformément ; en pratique dans les souffleries : le diamètre doit être 10x supérieurs à la section de l'objet que l'on teste
\item odg finesse : parapente 10, oiseau 10, ie 1m en horizontal descend de 10m
\item les avions ne battent pas des ailes car trop d'effort, on ne comprend pas tout des écoulements, interactions compliquées, petits drones oui
\end{itemize} \end{remarques}


%\part{Thermodynamique}

\include{LP06}
\Chapter{Transitions de phase}

\nivpre{L3}{
 \begin{itemize}
  \item potentiels thermodynamiques
  \item électromagnétisme dans les milieux
 \end{itemize}
}

\lemessage{Message important à faire passer lors de la leçon.}

\biblio{}

\section*{Introduction}
définition phase

\section{Transition liquide-vapeur d'un corps pur}
\subsection{Approche phénoménologique}

Description (P,T), (P,V) \\
Point critique, triple

\begin{ecran}
	Diagrammes thermodynamiques de l'eau
\end{ecran}

\subsection{Étude thermodynamique}
\subsection{Discontinuité et relation de Clapeyron}
\subsection{Mise en œuvre expérimentale}

\section{Transition paramagnétique-ferromagnétique}
\subsection{Modélisation}
\subsection{Minimisation du potentiel}



\begin{experience}
	Chauffer diazote liquide, peser et déterminer la masse perdue par vaporisation ; faire sans chauffer, puis en chauffant
\end{experience}



\section*{Conclusion}

H=f(M) pour différent T, quand T diminue en-dessous de température de Curie : bifurcation fourche


\begin{remarques} \begin{itemize}
\item point critique : opalescence
\item eau : liquide-vapeur pente (P,T) négative 
\item Landau : discussion de symétries, brisure de symétrie transition para-ferro, on privilégie un certain axe ; la phase la plus ordonnée est ferro (ordre magnétique)
\item liquide-vapeur : pas de brisure de symétrie car au-delà du point critique, passe continûment de l'un à l'autre
\item autres transitions de phase : métal supraconductrice vers conducteur (transition du premier ordre ?), condensat de Bose-Einstein superfluide, variétés allotropiques, cristaux liquides
\item état métastable : la création de surface entre deux phases a un coût, le changement d'état est retardé, un système dans un tel état peut être amené vers l'état stable à partir d'une petite perturbation (exemple : eau congélateur, sortir au bon moment, une pichenette et ça gèle instantanément)
\item autre exemple métastable : chevaux hiver 1942 Leningrad, lac Ladoga, congelés dans l'eau
\item liquide-vapeur métastable : nucléation dans casserole quand on fait chauffer l'eau : les bulles viennent d'endroits particuliers, nécessite défauts. Pression diminue tant qu’aucune bulle de vapeur n’apparaît pas. Pression peut être négative dans le liquide si pas de défaut paroi : la cohésion des molécules entre elles et avec les parois  permet d’exercer cette pression négative (terme $a/V^2$ dans van der waals pour fluide) (\href{https://www-liphy.univ-grenoble-alpes.fr/pagesperso/marmottant/Publications_files/ArticleFinalCouverturePlusArticle.pdf}{lien pour sources}). Les pressions négatives observées dans de l’eau peuvent atteindre plusieurs centaines de fois la pression atmosphérique ! Cavitation : défaut provoque la nucléation d'une bulle
\item solide-liquide métastable : brouillard givrant (eau en surfusion)
\item le diamant est métastable
\item chambre à bulles : hydrogène liquide maintenu dans un état surchauffé, champ magnétique, trajectoire particule courbée, matérialisée par formation d'une trainée de bulles (Prix Nobel 1960)
\item chambre à brouillard : inverse, trainée de condensation (Prix Nobel 1927, ancêtre ?) pour l'étude de particules radioactives, étude  de produits de réactions nucléaires, étude d'interactions... Charles Wilson l'a inventé en essayant de comprendre pourquoi les nuages se forment. Première méthode d'imagerie
\item cristaux liquides : transition smectique-nématique... vaste sujet chimique
\item le fer n'est pas un bon exemple de transition para-ferro car il y a aussi un changement de variété allotropique vers la température de Curie
\item brisure spontanée de symétrie généralisée par chapeau mexicain : modèle de Higgs
\end{itemize} \end{remarques}
\include{LP08}
\include{T1}
\include{T2}
\include{T3}
\include{T4}
\include{T5}
\Chapter{Entropie en physique statistique et thermodynamique}

\nivpre{L3}{
 \begin{itemize}
  \item 
  \item 
  \item 
 \end{itemize}
}

\lemessage{.}

\biblio{}

\section*{Introduction}

\section{}
\section{}
\section{}


\section*{Conclusion}



\begin{remarques} \begin{itemize} 
\item introduit en 1865 par Rudolf Clausius, un terme qui fait référence à la fois à l'énergie et à la transformation
\item l'entropie d'un corps noir est proportionnelle à son aire (et colossale), renseigne sur quantité d'information que renferme le trou noir ; lorsqu'un objet tombe dedans, le trou noir absorbe son entropie, ce qui permet à l'univers de pas avoir une entropie qui diminue
\item plasmas : convergent vers équilibre sans changement d'entropie, réversible grâce à interaction à distance (champ électrique) ; Landau prétendait que même sans collision, le plasma se rapprocherait de l’équilibre suite à une diminution du champ électrique (pou équation de Vlasov-Poisson linéarisé, Villani l'a démontré avec non-linéaire)
\item 
\item 
\item 
\item 
\item 
\item 
\end{itemize} \end{remarques}
\Chapter{Phénomènes de diffusion}

\nivpre{CPGE}{
 \begin{itemize}
  \item premier principe de la thermodynamique
  \item bilans locaux (d'énergie, ttc...)
 \end{itemize}
}

\lemessage{.}

\biblio{}

\section*{Introduction}
terme du langage courant, qu'est-ce que ça veut dire en physique ? \\
Vidéo diffusion : diffusion microscopique \\
application : isolation habitat, résistance thermique \\


\marginpar{réf. \cite{ToutenunPC_2016},\cite{Diu_thermo}}

\section{Diffusion de particules}
\subsection{}
Bilan \\
Diffusivité en $m^{2}.s^{-1}$
\subsection{Loi de Fick}
phénoménologique \\
irréversibilité \\
ex : diffusion parfum : $D\approx10^{-6}-10^{-4}$, $\tau=l^2/D=115\,jours$ à une distance de 10m \\
sucre dans eau : $D\approx10^{-12}-10^{-8}$, café $\tau=10^{11}s$ \\
phénomène très lent
\subsection{Équation de diffusion}


\subsection{Analogie thermique}
On a la même chose !! Tableau \\
phénomènes lents à nos échelles \\

\subsection{Diffusion thermique}
(pression constante car s'équilibre avec l'environnement, système est difficile à définir, éviter de dire des bêtises, utiliser H (voir \cite{Diu_thermo})ou juste analogie) \\
(conditions aux limites pas évidentes) \\

\subsubsection{Densité de courant thermique et équation de conservation}
Flux thermique \\
Établissement de l'équation de conservation 1D \\
\subsubsection{Loi de Fourier et équation de diffusion}
même chose avec diffusivité thermique \\
OdG conductivité thermique latériaux non métalliques $\sim 1\,W.s^{-1}.K^{-1}$, métaux $\sim 100\,W.s^{-1}.K^{-1}$
\subsubsection{En régime permanent : détermination de la conductivité thermique}
expérience avec cuivre \\
résultat : $460\,W.s^{-1}.K^{-1}$

\section{Aspects microscopiques}
Marche aléatoire (voir cours Alain Aspect) \\
relation d'Einstein \\

\section{Applications}
voir Termniale STI2D \\
résistance thermique \\
résistance hydraulique \\

\section*{Conclusion}
Tableau récapitulatif des deux diffusions \\
Généralisation \\
Plus tard : en mécanique des fluides diffusion de la quantité de mouvement \\


\begin{remarques} \begin{itemize} 
\item résistance thermique $\Delta T = R_{th} \Phi$, analogue à loi d'Ohm
\item dépend de la surface ou du contour de la surface ?
\item définir problème 1D : $j_Q$ ne dépend que d'une variable spatiale ; plusieurs dimensions : surface peut changer, étudier le rapport ? 
\item barre unidimensionnelle : il y a des pertes latérales, donc ne dépend pas d'une dimension spatiale
\item intéressant : calcul marche aléatoire pour établir diffusion
\item différence conductivité thermique métaux/non-métaux car conductivité électrique
\item renversement du temps : solution n'est plus la même, donc pas réversible
\item exemple de source de diffusion : réaction chimique, création d'une molécule ; réaction nucléaire (source de particules et de température) ; photon dans soleil
\item exemple où Fick ne marche pas : supernova, bombe nucléaire 
\item 
\item diffusion de matière, diffusion thermique, diffusion de quantité de mouvement, diffusion de charge
\item relation d'Einstein : fluctuation-dissipation
\item milieux poreux chez BCPST
\item exemples en biophysique : livres PACES (on a des livres d'exos pour médecine \cite{biophysique}), diffusion dans gel, quantité de nutriments dans cellule (taille limite d'une cellule)  
\item 
\item 
\item HS : diffusion des ondes EM (élastique vs inélastique) : onde-particule (Thomson, Compton), onde-matière (Rayleigh, Mie, Raman)
\end{itemize} \end{remarques}
\include{LPn21}
\include{LPn22}
\include{LPn37}
\include{LPn45}



%\part{Electromagnétisme}

\Chapter{Conversion de puissance électromécanique.}

\nivpre{CPGE}{
 \begin{itemize}
  \item induction
 \end{itemize}
}

\lemessage{Applications centrale dans la vie de tous les jours, s'applique majoritairement en France à deux reprises dans la chaîne de production/d'acheminement de l'électricité. Moteurs seulement au programme PSI.}

\biblio{}

\section*{Introduction}


Moteurs à courant continu \\

Application : \href{ge.com/content/dam/gepower-nuclear/global/en_US/documents/Nuclear-Product-Catalog.pdf}{paramètres génératrice d'EPR Flamanville} ou peut-être \href{https://stage.geready.com/content/dam/gepower-pw/global/en_US/documents/alstom/gea31902-nuclear-turbine-island-solutions-29-10-15.pdf}{ce lien}




\transition{.}



\begin{experience}
	Expérience pour illustrer
\end{experience}

\begin{attention}
	Erreur faite souvent ou point sur lequel insister.
\end{attention}


\begin{remarques}
	Remarque concernant le contenu
\end{remarques}
\include{LP10}
\include{LP22}
\include{LP23} 
\include{LP24}
\include{LP25}
\include{E1}
\Chapter{Le champ magnétique}

\nivpre{CPGE}{
 \begin{itemize}
  \item électrostatique (équations de Maxwell et champ électrostatique)
  \item théorèmes d'Ostrogradski et de Stokes
  \item coordonnées cylindriques
  \item définition de l'intensité
 \end{itemize}
}

\lemessage{.}

\biblio{}

\section*{Introduction}
Grandeur vectorielle \\
applications au quotidien : aimants \\
champ magnétique terrestre mesuré par boussoles \\

\section{Propriétés du champ magnétique}
\subsection{Sources}
1820 Oersted, physicien dannois, expérience : aiguille métallique à proximité d'un circuit électrique \\
une interprétation possible : source de champ magnétique = déplacement de charges, ie courant électrique \\
ordres de grandeurs... \\

\subsection{Flux du champ $\vec{B}$}
flux à travers surface fermée est nul \\
lignes de champ sont fermées \\
\subsection{Circulation du champ $\vec{B}$}
théorème d'Ampère \\

\section{Expression du champ magnétique}
\subsection{Symétries}
de la distribution de courant \\
à partir de la force de Lorentz \\
\subsection{Invariances}
\subsection{Application à un câble cylindrique}

\section{Induction électromagnétique}
schéma mouvement d'un aimant dans une spire : production d'un tension \\
Loi de Faraday \\
Loi de Lenz \\
Applications : microphone, surtout haut-parleur \\

\section*{Conclusion}
Champ magnétique terrestre : dipôle magnétique (sud au nord géographique, angle $11.5^o$ par rapport à l'axe de rotation de la Terre) \\
moteurs \\



\begin{remarques} \begin{itemize} 
\item $\div \vec{B} = 0$ car il n'y a pas de monopole magnétique ; si ça existait : th de Gauss, équivalent du champ électrostatique radial Coulomb
\item créer un champ magnétique qui simule un champ magnétique : solénoide semi-infini (très grande longueur, regardée à une distance intermédiaire) $\rightarrow$ champ radial
\item différence entre force de Lorentz et force de Laplace : la force de Lorentz ne travaille pas mais la force de Laplace peut travailler car force appliquée par les charges au conducteur, perpendiculaire au courant mais le déplacement est arbitraire   
\item cas où il y a une influence des charges sur le champ magnétique : charges en mouvement ; il faut aussi considérer leur vitesse pour connaître influence
\item cas d'une bobine carrée
\item cas d'une bobine torique
\item calcul champ spire sur l'axe
\item solénoïde
\item H historiquement a été introduit en premier
\item il faut absolument parler de l'aimant
\item penser aux analogies dipôle électrique/dipôle magnétique
\end{itemize} \end{remarques}
\include{LPn15}
\include{LPn20}

%\part{Ondes / électronique}

\include{LP11}
\include{LP12}
\include{LP13}
\Chapter{Ondes acoustiques}

\nivpre{CPGE}{
 \begin{itemize}
  \item A
 \end{itemize}
}

\lemessage{Message important à faire passer lors de la leçon.}

\biblio{}

\section*{Introduction}

\begin{ecran}
	Contenu affiché sur diapositives
\end{ecran}

\begin{remarques}
	Remarque concernant le contenu
\end{remarques}

\transition{Belle transition entre deux parties ou sous-parties.}



\begin{experience}
	Expérience pour illustrer
\end{experience}

\begin{attention}
	Erreur faite souvent ou point sur lequel insister.
\end{attention}
\include{LP15}
\include{On1}
\include{On2}
\include{On3}
\Chapter{Équations d'onde}

\nivpre{CPGE}{
 \begin{itemize}
  \item électrocinétique : loi de Kirchoff
  \item électromagnétisme : eq de Maxwell, force de Lorentz
  \item équation de d'Alembert (établi dans le cas de la corde 1D par exemple)
  \item dérivée particulaire
 \end{itemize}
}

\lemessage{Ondes dans plein de domaines.}

\biblio{}

\section*{Introduction}
définition onde et équation d'onde \\
propagation sans transport moyen de matière dont les dépendances sont définies par une équation d'onde \\
équation d'onde : équation différentielle à dérivées partielles, établit une relation entre dépendance temporelle et dépendance spatiale  \\
\marginpar{\cite{ToutenunMP_2004}}

autres : \cite{olivier_ondes}, \cite{guyon_hydrodynamique}

\section{Milieu non dispersif}
\subsection{Câble coaxial}
établissement de l'équation de d'Alembert \\
prendre le temps d'expliquer le modèle, dimensions infinitésimales \\

\subsection{Solutions de l'équation}
def surface d'onde, les différentes solutions : planes, sphériques, progressives \\
\subsection{Équation de dispersion}
vitesse de phase, vitesse de groupe \\


\section{Milieu dispersif}
définition atténuation, absorption \\
dispersif=vitesse de phase dépend de omega, il n'y a pas forcément d'absorption \\
retour sur le cable coax avec résistances (équation des télégraphistes)  \\

\subsection{Équation de Klein-Gordon}
\marginpar{\cite{ToutenunMP_2004} p516}
\marginpar{\cite{ToutenunPC_2016} p1001}

modèle du plasma dilué (néglige interactions entre particules du plasma) non relativiste \\
gaz ionisé neutre, hypothèse vitesse ions négligeable car beaucoup plus lourd \\
on regarde en fait la moyenne des électrons : PFD sur un électron moyen \\
par exemple ionosphère \\

\subsection{Relation de dispersion}
vitesse de phase, vitesse de groupe \\

\section*{Conclusion}
aussi en quantique, équation de Schrödinger \\
équations non linéaire \\
solitons \\



\begin{remarques} \begin{itemize} 
\item une équation d'onde n'est pas forcément linéaire
\item onde évanescente : pas qu'en MQ, ne se propage pas, SPP le long de la pénétration, miroir à atome
\item Kichoff : nécessite ARQS (propagation des ondes dans le milieu plus rapide que variation de courant et tension, dépend de la taille du système)
\item attention Klein-Gordon c'est limite prépa
\item onde de surface : sujet riche
\item penser aux conditions aux limites pour le guidage : change modes de propagation
\item plan alternatif : I/Onde longitudinale dans plasma (1) Modèle plasma, 2) établir eq d'onde, 3) eq de Klein-Gordon), II/Onde de surface (le guyon p258 (combiner Euler, évolution isentropique, conservation...), TF -> eq de dispersion avec tanh, beaucoup de régimes limites, plein de choses à dire, \href{https://www.youtube.com/watch?v=95sQcSulRFM}{vidéo Youtube forme des ondes de surface}...), ccl: ondes solitaires (phénomène non-linéaire)
\item équation télégraphiste, corde avec rigidité (dérivée 4eme de la position apparaît), onde acoustique
\item Schrödinger n'est pas irréversible ! Vraie pour l'équation de la chaleur
\item autre équation d'onde : équation thermique (onde de chaleur)
\item phénomènes non-linéaires : chaîne d'oscillateurs, pendule pesant...
\end{itemize} \end{remarques}

	
\include{LPn01}
\Chapter{Ondes évanescentes}

\nivpre{CPGE 2ème année}{
 \begin{itemize}
  \item électromagnétisme (tout le cours)
  \item mécanique quantique (équation de Schrödinger)
  \item optique géométrique
 \end{itemize}
}

\lemessage{Vu dans plusieurs domaines, ici on fait le lien entre tout, transversal.}

\biblio{}

\section*{Introduction}
qqch qui s'atténue très rapidement dans une zone spatiale \\
définition propre... (nécessiter de sonder pour observer l'existence d'une telle onde) \\
manip : laser He-Ne, dioptre hémisphérique (ENSP585) (propriété : l'incidence est toujours normale), réflexion interne totale à l'interface rectiligne : a priori il n'y a pas d'onde, mais si l'on prend un autre prisme hemisphérique que l'on place approximativement proche de l'autre, on a un rayon, donc il y a transmission \\
ça s'appelle réflexion interne frustrée \\ 
\href{https://fr.coursera.org/lecture/mecanique-quantique/8-2-la-reflexion-totale-interne-frustree-e5hyh}{Vidéo IOGS Manuel Joffre}

\section{Ondes évanescentes électromagnétiques}
\subsection{Définition et mise en évidence expérimentale}
\subsection{Explication théorique : dépasser les lois de Snell-Descartes}
expression angle réflexion totale interne théorique\\
Equation de Helmholtz (attention c'est bien dans l'espace de Fourier), diélectrique homogène uniforme\\
étude réfraction en EM : k parallèle à l'interface est identique, omega aussi, k perpendiculaire peut être complexe
\section{Ondes évanescentes au sens large}
\subsection{Élargir à d'autres domaines de la physique}
effet de peau, plasma
autres exemples en physique (pas sûr):  \\
ondes évanescentes thermiques \\
dissipation de la quantité de mouvement dans un fluide visqueux \\
\subsection{Ondes évanescentes de matière : effet tunnel}
marhce de potentiel \\
résolution dans les 3 zones \\
\section{Applications}
\subsection{Microscopie en champ proche}
SNOM \\
sonde : guide d'onde \\
\subsection{Microscope à effet tunnel}
STM \\
\subsection{Détection d'empreintes digitales}
réflexion interne frustrée \\
(également : couplage entre guides d'onde, détecteur de rosée...cf mon projet ETI IOGS)


\section*{Conclusion}
partout, application technologique intéressante

\begin{remarques} \begin{itemize} 
\item \href{https://hal.archives-ouvertes.fr/tel-02493813/document}{thèse de Claire Li} \cite{ClaireLi}, chap2
\item plasmon de surface et plasmon de volume : filtre, couleurs et compagnie..
\item imagerie par résonance de plasmon...
\item le test de grossesse : première application bioplasmonique (voir slides Jerome Wenger)
\item capteur à résonance plasmon de surface (configuraiton de Otto ou Kretschmann)
\item traitement photothermique du cancer
\item évanescent ne veut pas toujours dire décroissance exponentiel : exemple avec effet tunnel et barrière par rectangulaire (WKB...) ; autre exemple barrière coulombienne...
\item déplacement de Goos-Hiinchen (TD d'électromagnétisme de Jeremy Neveu)
\item effet de peau : dissipation, sinon pas de dissipation
\end{itemize} \end{remarques}
\Chapter{Oscillateur harmonique : approximation et limitations}

\nivpre{CPGE}{
 \begin{itemize}
  \item PFD
  \item loi des mailles
  \item équations différentielles
 \end{itemize}
}

\lemessage{L'OH c'est partout, c'est la vie. Points importants : conservation énergie mécanique, équipartition de l'énergie, isochronisme des oscillations, oscillations sinusoïdales perpétuelles.}

\biblio{}

\section*{Introduction}
Comment est défini le temps ? oscillateur naturel \\

%\marginpar{réf. \cite{ToutenunPC_2016},\cite{hecht},\cite{huard_pola}}

\section{Étude d'un oscillateur harmonique}
\subsection{Observations}
expérience : masse sur coussin d'air avec 2 ressorts \\
beau schéma de l'expérience \\

\subsection{Mise en équation}
PFD à m dans R galiléen projeté sur l'axe Ox \\
Force de droite $\vec{F}_d$
Force de gauche $\vec{F}_g$
ED $\ddot{x}+{\omega_0}^{2} x = 0$, $\omega_0=\sqrt{\frac{2k}{m}}$
solution $x(t)=A \cos \omega_0 t + B \sin \omega_0 t$
conditions initiales : $x(0)=x_0$, $\dot{x}_0=0$
représentation graphique $x=f(t)$, oscillations, période \\

\subsection{Énergie}
cinétique et potentielle \\
définition énergie potentielle élastique $\vec{F}_{ressort} \cdot \vec{e}_x = -\frac{\partial E_{pot}}{\partial x}$  \\
$E_{pot}=2 \times \frac{1}{2} k x^2 $ \\
$E_c=\frac{1}{2} m \dot{x}^2$, moyenne : $<E_c>=\frac{1}{2} k {x_{0}}^2=<E_{pot}>$ \\
équipartition de l'énergie \\
présentation portrait de phase (normalisé) : cercle, sens \\
acquisition de la position sur logiciel Tracker \\
tracé portrait de phase : on observe une spirale \\
transition \\

\section{Approximations et limitations}
\subsection{Oscillateur harmonique amorti}
comment prendre en compte cet amortissement \\
Prise en compte d'une force supplémentaire : force de frottement fluide $\vec{F}_{frott}=-\alpha \vec{v} = -\alpha \dot{x} \vec{e}_x$ \\
équation différentielle : $\ddot{x}+\frac{\omega_0}{Q} \dot{x} +{\omega_0}^{2} x = 0$, $\omega_0=\sqrt{\frac{2k}{m}}$ \\
 facteur de qualité $Q=\frac{m \omega_0}{\alpha}$\\
 résolution avec exponentielle complexe, racines de l'équation caractéristique... \\
 $x(t)=x_0 \exp{-\frac{\omega_0}{2Q}t} \cos{\omega_0 t + \varphi} $ \\
 On voit le rôle du facteur de qualité dans les oscillations \\

\subsection{Amplitude des oscillations} 
pendule simple \\
comportement similaire \\
mesure période : amplitude faible vs amplitude forte \\
1 seconde d'écart, il y a une influence : surprenant, différent d'avant \\
Etude : résolution avec l'énergie \\
approximation $\sin{\theta} \approx \theta $ \\
$\ddot{\theta} + {\omega_0}^2 \theta =0$  avec $\omega_0=\sqrt{\frac{g}{l}}$ \\
isochronisme des oscillations en théorie, ne dépend pas de l'amplitude : bizarre $\rightarrow$ un autre modèle \\
en fait l'énergie potentielle est en cosinus, on l'a approché par une parabole \\
allure $\frac{T}{T_0}=f(\theta_0)$ (parle pas de la formule de Borda) \\
portrait de phase : n'est plus circulaire, va être en forme d'ellipse \\
perte d'isochronisme des oscillations \\
approximation non vérifiée \\


\section*{Conclusion}
diapason, oscillateur à quartz fonctionne pareil, oscillation des bras, s'atténue très peu dans le temps \\
mais facteur de qualité : $10^6$ \\
pour améliorer : utilisation de résistance négative avec condensateur \\

\begin{remarques} \begin{itemize} 
\item trop de temps sur les bases, mettre masse-ressort en prérequis
\item faire l'amortissement en slide, vite fait, détails en TD
\item écrire tous les messages importants, conditions initiales, notation x point...
\item oscillations autour d'un puits de potentiel : très important, le faire plus tôt
\item niveau : début de première année de CPGE
\item si un seul ressort : cette manipulation ne marche pas, il faut contraindre la masse mais ne pas rajouter de frottement, donc on choisit de mettre deux ressorts, ne change rien, juste valeur de la pulsation, position d'équilibre ne va pas être au milieu
\item il faut dire que l'oscillateur harmonique est un modèle idéal, mathématique, c'est définit par cette équation différentielle
\item vraie pour bcp de situations physiques : au voisinage d'une position d'équilibre (dérivée énergie potentielle est nulle), ne marche pas lors de bifurcation fourche par exemple (dérivée seconde de l'énergie potentielle est nulle)
\item énergie potentielle élastique : énergie stockée dans le ressort
\item OH à plusieurs dimensions ?
\item équipartition vraie tout le temps ? fait référence à quelque chose ? $\frac{1}{2} k_B T$ par degré de liberté
\item utilité du portrait de phase : comprendre les échanges énergétiques dans le système
\item espace des phases : formalisme hamiltonien, espace des coordonnées $q_i$ et impulsions $p_i$
\item pourquoi frottement fluide ? air visqueux sous la masse, dans le coussin d'air, si nombre de Reynolds faible dans cette couche
\item autres dissipations possibles : ressort, ou liaisons, potence... 
\item trajectoire dans le cas d'un frottement solide ? ce ne serait plus sinusoïdal
\item formule de Borda ; si on veut aller plus loin ? terme suivant en ${theta_0}^4$ car parité (symétrie pendule)
\item contre-exemple parité : force newtonienne autour de sa position d'équilibre
\item différents régimes : apériodique, pseudo-périodique
\item résumé : conservation énergie mécanique, équipartition de l'énergie, isochronisme des oscillations, oscillations sinusoïdales perpétuelles
\item quantique : énergie quantifiée $E_n=(n+\frac{1}{2}) \hbar \omega_0$, valeurs propres du hamiltonien, énergie minimale $\frac{\hbar \omega_0}{2}$
\item classique : varie entre 0 et énergie mécanique
\end{itemize} \end{remarques}
\include{LPn42}
\Chapter{Effet Doppler et applications}

\nivpre{Licence}{
 \begin{itemize}
  \item cinématique relativiste
   \item mécanique classique
   \item optique ondulatoire
 \end{itemize}
}

\lemessage{Tout type d'ondes. Sirènes qui s'approche/s'éloigne, radar voiture et redshift des étoiles qui s'éloignent.}

\biblio{}

\section*{Introduction}
prononcer Christiane Doppler (autrichien) \\
fréquence du son détermine sa hauteur (grave/aigu) \\
émetteur en mouvement : son différent \\
on le voit avec sirène qui se déplace \\
parler de monsieur Doppler, proposer mais faux car pas rélativité prise en compte \\
on va le traiter quantitativement \\


%\marginpar{réf. \cite{ToutenunPC_2016},\cite{hecht},\cite{huard_pola}}

\section{Effet Doppler classique}
$S$ : source d'ondes à la fréquence $\nu_0$, vitesse $v_S$ \\
$R$ : récepteur, reçoit à $\nu_1$, vitesse $v_R$ \\
$c$ : célérité des ondes \\
$\vec{u}$ : vecteur unitaire entre S et R

\subsection{Source mobile, récepteur fixe}
première impulsion : $t_1'=t_1 + \frac{d_1}{c}$ \\
deuxième impulsion : $t_2'=t_2 + \frac{d_2}{c}$ \\
$T_0 = t_2 - t_1$ et $T'=t_2'-t_1'$ : $T'=T_0 + \frac{d_2-d_1}{c}$ \\
calcul ... \\
hyp : $\vec{u}$ ne bouge pas trop \\

$f'=f_0 \frac{1}{1-\frac{\vec{v}_S \cdot \vec{u}}{c}}$ \\

Source s'éloigne de R : $f'<f_0$ : le son est plus grave \\
Source s'approche de R : $f'>f_0$ : le son est plus aigu \\

animation internet : un véhicule émet des ondes en se déplaçant \\
à l'avant du véhicule : fronts d'onde sont plus resserrés vers l'avant (aigu, longueur d'onde plus faible), plus éloignés vers l'arrière (grave)

\subsection{Source fixe, récepteur mobile}

Astuce : changement de référentiel \\
$f'=f_0 \left[ 1-\frac{\vec{v}_R \cdot \vec{u}}{c} \right]$ \\

\subsection{Formulation générale}
$f'=f_0 \frac{c-\vec{v}_R \cdot \vec{u}}{c-\vec{v}_S \cdot \vec{u}}$ \\
décalage Doppler $\frac{\Delta f}{f_0}$\\

vitesse du son, avion supersonique, formation d'un cône de front d'onde, énergie s'accumule, onde de choc \\

\section{Effet Doppler relativiste} 

transformée de Lorentz \\
étudier $\omega$ et $\vec{k}$
$f'=f_0 \sqrt{\frac{c-v}{c+v}}$ \\ 
Formule de Doppler-Fizeau : $f'=f_0 \gamma \left( 1-\frac{\vec{v}}{c} \cdot \vec{u} \right)$ \\
si la vitesse est perpendiculaire à l'axe $\vec{u}$, $f'=\gamma f_0 > f_0$ 



\section{Applications}
\subsection{Mesure de vitesse : radar}
détection hétérodyne  \\
signal reçu : $\omega_r$ \\
signal émis : $\omega_s$ \\
multiplieur : on obtient $\omega_r-\omega_s$ et $\omega_r+\omega_s$ \\
filtre passe-bas : on garde que le signal $\omega_r-\omega_s$
expérience

\subsection{Élargissement spectral}



\section*{Conclusion}

décalage longueur d'onde d'une source lumineuse vers le rouge quand elle s'éloigne \\
application Doppler en médecine : radio qui mesure la vitesse du sang dans vaisseaux sanguins, ultrasons \\


\begin{remarques} \begin{itemize} 
\item obligé de le mettre en niveau L3 et de faire relativiste
\item l'expérience est incontournable
\item ODG
\item important : notion d'invariance, on change de référentiel, on retrouve la même physique, principe de relativité
\item vitesse relative permet de ne traiter qu'un sens
\item physiquement, sont-ce deux cas différents ? non, l'un ou l'autre est immobile, le référentiel est galiléen ou non, sans importance, c'est la vitesse d'entraînement de l'un par rapport à l'autre
\item référentiel : ensemble de coordonnées d'espace-temps qui ont le même temps
\item être clair sur la définition des vitesses, référentiels, hypothèses
\item onde de choc en hydrodynamique : discontinuité de la pression qui se propage, voir la théorie
\item décalage vers le rouge : contexte astrophysique non cosmologique
\item dilatation des temps : démo avec un photon qui est réfléchi et qu'on "récupère" avec une vitesse de l'émetteur/récepteur
\item pendant une période, source bouge
\item faire modèle Newtonien de l'univers en expansion ?
\item référentiel d'inertie : seule configuration où omega et k sont différents est l'effet Doppler 
\item référentiel accéléré : le nombre de particules détectées (étude quantique) dépend du référentiel, à la base du rayonnement thermique des trous noir
\item effet Doppler sur ondes de matière : ondes sismiques, voir application
\item radar
\item Doppler application en médecine
\end{itemize} \end{remarques}
\include{LPn29}
\Chapter{Régulation et asservissement}

\nivpre{L3}{
 \begin{itemize}
\item principe des moteurs à courant continu
\item transformée de Laplace 
\item fonction de transfert
\item amplificateur opérationnel
\item diagramme de Bode
 \end{itemize}
}

\lemessage{.}

\biblio{}

\section*{Introduction}
on fait en permanence de la rétroaction dans la vie de tous les jours (quand on conduit...) \\
régulation : suivre une consigne constante quand il y a des perturbations \\
asservissement : suivre une consigne qui peut changer \\

\marginpar{réf. \cite{Hprepa_electronique},\cite{Neveu_elec},\cite{TecDoc_PSI}}

\section{Commande d'un système linéaire et nécessité d'une rétroaction}
\subsection{}
\subsection{}

$H_{FTBO}=\frac{retour}{entrée}=A \beta $ \\
$H_{FTBF}=\frac{sortie}{entrée}=\frac{A}{1+H_{FTBO}}$ \\

\section{Application : asservissement en vitesse d'un moteur}
expérience : boîtier moteur à courant continu MCC\\
rétroaction : capteur dynamo \\
fonctions de transfert \\
2 entrées \\

\subsection{Principe du moteur}
\subsection{Stabilité}
critères \\
\subsection{Précision}
théorème de la valeur finale \\
\subsection{Rapidité}

AO : conservation du produit gain-bande (inutile ici) \\


\section{Correction}
proportionnel, intégrale, dérivateur \\


\section*{Conclusion}
oscillateur : pas d'entrée, addition \\
systèmes biologiques \\
laser \\

\begin{remarques} \begin{itemize} 
\item ces notions ne sont pas au programme de CPGE
\item faire pas beaucoup d'exemples mais très bien les faire, aller chercher les vieux livres de prépas PSI
\item ne pas hésiter à faire des choses simples, pas faire trop de formules
\item régulation : cas particulier de l'asservissement
\item commande en position vs commande en vitesse
\item 
\item thermostat, perturbation : ouvrir une fenêtre en hiver par exemple
\item qu'est ce qui fournit l'énergie ? toutes les alimentations extérieures au circuit (AO, tension de commande)
\item électronique de puissance vs électronique de signal : puissance à l'amplificateur, tout le reste signal ; électronique de signal est plus précise que électronique de puissance
\item si alimentation moteur pas précise, pas grave car relatif, si pb dynamo (moins précise par exemple), plus grave car pas la même valeur mesurée, seuils pas les mêmes, moins bien régulé ?
\item on se fout de la précision dans chaîne directe, pas dans la rétroaction
\item exemple perceuse, pas besoin asservi car s'auto-régule ? on cherche une vitesse de rotation constante, donc régulation, automatisé pour avec une adaptation plus rapide, et plus de précision
\item soustracteur : AO en amplificateur non-inverseur
\item comment transformée de Laplace peut s'intégrer dans le programme CPGE ? pourquoi choix L3 ? car il faut transformée de Laplace
\item fonctions de transfert, hypothèses : système linéaire et vrai à tout amplitude
\item pas différent de fonction de transfert avec AO, même résultat mais schéma-bloc : modèle pour simplifier 
\item qu'apporte la conservation du produit gain-bande ? voir que la bande-passante s'agrandit si on a une rétroaction ; et impact sur la rapidité
\item en réalité, alimentation limite la tension de l'oscillateur à pont de Wien, saturation
\item notion de stabilité : dans la pratique, on a toujours des choses qui vont limiter
\item marges pour la sécurité : marges de gain et de phase ; voir avec diagramme de Bode (gain inférieur à 1 en phase)
\item étude du moteur en précision : commande en position ou vitesse ? 
\end{itemize} \end{remarques}
\Chapter{Inductances de l'électromagnétisme à l'électrocinétique}

\nivpre{CPGE}{
 \begin{itemize}
 \item théorème d'Ampère
 \item loi de Lenz, loi de Faraday
 \item bases de l'électrocinétique (lois des mailles, noeuds)
 \item utilisation de complexes en EC
 \end{itemize}
}

\lemessage{.}

\biblio{}

\section*{Introduction}
il existe des forces électromotrices induites, couplage avec le champ magnétique \\
comment ça marche dans un circuit ? cadre de validité ? \\
\marginpar{réf. \cite{Neveu_elec},\cite{ToutenunPSI_2016}, \cite{Perez_EM}, \cite{BFR_EM_4}}

\section{Rappel des lois de l'induction}
\subsection{Notions de flux et d'inductance}
flux $\Phi=...$ \\
on définit l'inductance par $L=\frac{\Phi}{i}$ \\
\subsection{Loi de Faraday}
force électromotrice induite $fem=e=-\frac{d\Phi}{dt}$ \\
alors $e=-L \frac{di}{dt}$ car L est constant \\
on retrouve loi classique de l'électrocinétique
\section{Inductance propre/mutuelle}
\subsection{Inductance propre}
Étude d'un solénoïde, théorème d'Ampère \\
$\Phi=\mu_0 \frac{N^2 S}{l}i$ \\
donc $L=\mu_0 \frac{N^2 S}{l}$ \\
ne dépend pas du temps tant que le circuit ne bouge pas, est indéformable  \\

\subsection{Inductance mutuelle}
il n'y a pas d'aimant, seulement le circuit, avec un autre : on a 2 solénoïdes \\
représentation des lignes de champ de chacun des solénoïdes \\
$M=\frac{\Phi_{1 \rightarrow 2}}{i_1}$ \\
Pour 2 solénoïdes imbriqués de même surface : $M=\mu_0 \frac{N_1 N_2}{l}$
\subsection{Passage à l'électrocinétique : schéma électrique équivalent}
pour une inductance propre : convention générateur et convention récepteur \\
la fem est définie en convention générateur \\
Pour 2 circuits couplés par inductance mutuelle : $U_1=L_1 \frac{di_1}{dt} + M \frac{di_2}{dt}$,...

\section{Utilisation en électrocinétique}
\subsection{Mesure de L par un circuit RL}
on a une petite résistance interne, que l'on néglige \\
$E=L \frac{di}{dt}+R i$ \\
réponse à un créneau, aux bornes de la résistance : $u(t)=E \left( 1-e^{-\frac{t}{\tau}} \right) $ avec $\tau = \frac{L}{R}$ \\
continu, faible variation donc passe-bas aux bornes de la résistance \\

expérience, décrément logarithmique : $0.63 E$ \\

\subsection{Mesure de M}
$u_1=L \frac{di_1}{dt}+M \frac{di_2}{dt}$ \\
$u_2=L \frac{di_2}{dt}+M \frac{di_1}{dt}$ \\
on prend une résistance très grande dans le circuit secondaire pour négliger $i_2$ \\
$\frac{u_2}{u_1}=\frac{M}{L}$


\subsection{Transformateur}
2 bobines traversées par le même champ B \\
en pratique : il y a un support (ferro) qui permet de canaliser les lignes de champ \\
rapport de transformation : $m=\frac{N_2}{N_1}=\frac{U_2}{U_1}$ \\
expérience : étude d'un transformateur 

\subsection{Circuit RL en régime sinusoïdal forcé} 
sortie aux bornes de la bobine \\
filtre passe-haut \\


\section*{Conclusion}
applications : filtrage, effet présent dans câble coaxial \\

\begin{remarques} \begin{itemize} 
\item toujours orienter les contours, lignes de champ
\item définition pulsation de coupure
\item si pas de variation de courant, on n'a pas d'induction, loi de Faraday
\item tu as fait circuit filiforme (1D), comment ça marche pour un circuit non-filiforme ? (2D)
\item solénoïde infini de taille fini : il faut que longueur soit très grande devant le rayon
\item l'auto-inductance est toujours positive car orientations sont définies comme ça
\item l'inductance propre ne dépend pas du métal, ne dépend pas de la conductivité électrique
\item l'inductance mutuelle de 1 sur 2 et de 2 sur 1, c'est pareil ?  oui, il faudrait le calculer avec Biot-et-Savart
\item inductance mutuelle : dépend de géométrie, matériau entre les 2, peut être négatif, dépend comment on oriente les circuits les uns par rapports aux autres
\item $K=\frac{M}{\sqrt{L_1 L_2}}=1$ pour bobines en influence totale (une ligne de champ passant dans une bobine passe aussi dans l'autre)
\item $abs(K)<=1$, dem : $E_{tot}=\frac{1}{2} \left( i_1 \phi_1 +i_2 \Phi_2 \right)=\frac{1}{2}\left( L_1 i_1^2  +L_2 i_2^2 + 2M i_1 i_2 \right)$, $E_{int}=M i_1 i_2$, et $E_{tot}=\frac{\left( \vec{B_1} + \vec{B_2}\right)^2}{2\mu_0} V$ donc $M^2<L_1 L_2$
\item induction de Neumann ou Lorenz ? Neumann, circuit fixe dans un champ variable
\item circuit filiforme $E=\frac{1}{2} \Phi i$
\item reconnaître un passe-bas, un passe-haut à l'allure de la réponse à un échelon
\item des formules de réciprocité...
\item montrer que champ magnétique est nul à l'extérieur : voir TD Erwan
\item voir L non contant avec géométrie variable : force d'un circuit sur lui même (tendance à s'ouvrir, s'éloigner), force créée par le solénoïde a tendance à l'écraser lui même ; action magnétique subie par le courant, force subie par une spire, règle du flux maximal : énergie potentielle associée aux forces de Laplace circuit se déplace de façon à maximiser $i \Phi$, variation positive de $L$
\item autre piste : quelle nouvelle vision peut nous donner l'électromagnétisme sur la bobine en électrocinétique ?
\end{itemize} \end{remarques}


%\part{Optique}

\Chapter{Microscopies optiques}

\nivpre{L3}{
 \begin{itemize}
\item optique géométrique
\item diffraction (critère de Rayleigh)
\item fluorescence
 \end{itemize}
}

\lemessage{Message important à faire passer lors de la leçon.}

\biblio{}

\section*{Introduction}
Microscope : appareil permettant d’étudier des détails microscopiques à une échelle macroscopique.
Petit historique : utilisation systématique de lentilles grossissantes à la fin du XVIe siècle. On veut : bon grossissement, bonne résolution, image fidèle (pas de défauts).

\section{Microscope classique}
\subsection{Dispositif}
\marginpar{\cite{Optique_houard} p.154}
\marginpar{\cite{app_miscroscope}}
\subsection{Grossissement commercial}

\subsection{Limites}
résolution : diffraction, critère de Rayleigh \\
aberrations des lentilles : correction des aberrations chromatiques avec verre de champ et verre d’oeil  \marginpar{\cite{Optique_houard} p.161}

\transition{Biologie échantillons transparents non colorés : compliqué.}

\section{Microscopie par contraste de phase}

\marginpar{\cite{constrastephase_kastler}}
\marginpar{\cite{phasecontrast_microscopy}}
\href{http://toutestquantique.fr/champ-sombre-et-contraste/}{Vidéo Youtube Tout est quantique sur le champ sombre et contraste}


Objet de phase : $s_0=V_0 \exp(i\omega t)$ et $s=s_0 \exp(i\varphi)$
\begin{itemize}
\item Strioscopie : on coupe $s_0$, donc on obtient $I=I_0 \varphi^2$. Problèmes : contraste très faible
\item Contraste de phase : $s_0$ retardé de $-\pi/2$ (lame de verre d’épaisseur $n\lambda+\lambda/4$ par exemple), on
obtient $I \approx I_0(1-2\varphi)$. Alors $C=\abs{2\varphi}$. Prix Nobel 1953 Zernike
\end{itemize}

Avantages : méthode non destructive, non intrusive. On peut observer du vivant!  \\

\transition{Observer parties spécifiques ou échantillon épais : 3D}

\section{Microscope confocale laser à fluorescence}
\marginpar{\cite{cohard_confocale}}
\marginpar{\cite{fluorescence_microscopy}}
Fluorescence : fluorophores dans l’échantillon (qui se fixent spécifiquement).  \\
Confocale laser : image point par point \\
Importance du miroir dichroïque. C’est là l’intérêt de la fluorescence : on peut facilement éliminer la lumière diffusée. \\


\section*{Conclusion}
méthodes non optiques : STM, AFM, MEB, TEM \\
résolution nanométrique et sub


\begin{remarques} Détails sur le microscope à contraste de phase
\begin{itemize}
\item Éclairage de Köhler : il permet d’éclairer l’échantillon de manière uniforme, et de ne pas avoir le filament superposé à l’échantillon (ce que l’on fait habituellement pour avoir la meilleur luminosité). Solution : faire l’image de la source sur le diaphragme d’ouverture du condenseur (ici le "condenser annulus"), qui est dans le plan focal objet du condenseur.
\item L'anneau de phase est placé dans le plan focal image de l’objectif, et c’est bien le plan de Fourier (on a bien un éclairage parallèle sur l’échantillon). Ce système conjugue ainsi le "condenser annulus" et le "phase plate". Par ailleurs l’objectif fait l’image de l’échantillon sur l’écran.
\item Pour bien comprendre le schéma, il faut voir que les traits sont des rayons (le coloriage entre traits est en fait plus perturbant qu’autre chose), et lorsqu’ils se recoupent on peut voir où est formée l’image de tel ou tel objet.
\end{itemize}
\end{remarques}


%\begin{ecran}
%	Contenu affiché sur diapositives
%\end{ecran}

%\begin{remarques}
%	Remarque concernant le contenu
%\end{remarques}

%\transition{Belle transition entre deux parties ou sous-parties.}



%\begin{experience}
%	Expérience pour illustrer
%\end{experience}

%\begin{attention}
%	Erreur faite souvent ou point sur lequel insister.
%\end{attention}
\include{LP17}
\include{LP18}
\include{LP19} 
\include{LP20} 
\include{LP21}
\Chapter{Laser}

\nivpre{L3}{
 \begin{itemize}
 \item Oscillateurs électroniques
 \item Equation de Schrödinger
 \item Electromagnétisme dans les milieux diélectriques
 \item Interféromètres de Michelson et Fabry Pérot
 \item Effet Doppler
 \item Forme des orbitales atomiques
 \end{itemize}
}

\lemessage{..}

\biblio{}

\section*{Introduction}

LASER : Light Amplification by Stimulated Emission of Radiation \\
Source lumineuse omniprésente et très utile : Cohérente, monochromatique, faisceau parallèle et concentré

\marginpar{\cite{ILM_Faroux}}
\marginpar{\cite{Optique_houard} ch11}
\marginpar{\cite{Optique_sextant} ch4}
\marginpar{\cite{laser_dangoisse}}
\marginpar{\cite{ToutenunPC_2016} ch30-31}


\section{Principe de fonctionnement}
\subsection{Rappel oscillateur électronique}
Condition de Barkhausen
\subsection{Principe du Laser}

\section{Amplification par émission stimulée}
\subsection{Système à deux niveaux}
Hyp : 2 niveaux d’énergies non dégénérés (a, b) interagissent avec le champ électromagnétique
Fonction d’onde du système : ...

\subsection{Coefficients d’Einstein. Emission stimulée}
Grâce au développement de la partie précédente nous pouvons obtenir l’équation d’Einstein \\
Condition pour avoir émission stimulée ...
\subsection{Pompage}
\section{Cavité résonnante}
\subsection{Cavité Fabry Pérot}
\subsection{Application : le LIDAR}
exploite l’effet doppler
\subsection{Le Laser stationnaire}
Point de fonctionnement du laser et raies transmissent par le laser...

\section*{Conclusion}
Laser : oscillateur optique \\
Amplification par émission stimulée découle de l’eq de Schrödinger \\
Filtrage peut être effectué par une cavité Fabry-Pérot \\
Ouverture sur les lasers semi-conducteurs \\


\begin{remarques}
\begin{itemize}
\item nb de pics et largeur de raies typique d’un laser : dépend de la largeur du gain et de la largeur des pics issus du Fabry-Pérot $10^10 Hz$ à $10^15 Hz$. On peut mettre un filtre pour sélectionner le mode qui nous intéresse (filtre de lLyot) et rendre aussi le laser monomode. Pour rendre un laser monomode on peut aussi jouer sur la taille de la cavité
\item taille cavité FP laser semiconducteurs : ordre du mm au μm (rq pour laser He-Ne de l’ordre de 17cm). Pour ces lasers le gain est plus relativement homogène, il est plus facilement monomode
\item laser à solide : milieu = cristal (par ex Al2O3) les électrons sont excités, pertes non radiatives dans le réseau cristallins puis désexcitation. Le gain est plus homogène. Le gain sature sur l’ensemble de la courbe et un seul mode lase. Pour un laser à gaz, le gain est inhomogène et le gain va saturer pour chaque groupe d’atomes faisant partie de la même classe de vitesse. Plusieurs modes vont pouvoir laser. Le plus connu Nd:YAG (ophtalmologie, nettoyage de façade cf : c’est pas sorcier)
\item laser les plus connus et applications : He-Ne, Argon, Xénon, Crypton, Diode laser (lecteur CD), semi-conducteur (utilisés dans toutes les télécommunications, fibre optique, téléphone, internet…). Rq : les lasers à gaz ne sont pas les plus utilisés, juste pour l’enseignement et en métrologie car ils sont plus stable. Mesure distance Terre-Lune, LIDAR, Correction de la myopie
\item laser à impulsion : on apporte de l’énergie en permanence (le gain augmente, mais on s’arrange pour qu’il y
ait plus de pertes que de gain). En pratique on peut utiliser des cellules de pockels dont on peut contrôler l’indice optique et la biréfringence avec la tension appliquée. en les associant avec un polariseur à $45^o$ des axes neutres, on peut en faire des interrupteurs contrôlés électriquement. On arrive ainsi à obtenir une très forte inversion de population et d’emmagasiner de l’énergie dans le milieu à gain. On commute les pertes $\rightarrow$ Tres forte inversion de population pour ramener le gain = pertes, libère bcp de photon
\item revoir gain en fonctionnement stationnaire, gain quand I non nul, gain en fonction de I
\end{itemize}
\end{remarques}

%
%\begin{ecran}
%	Contenu affiché sur diapositives
%\end{ecran}
%

%
%\transition{Belle transition entre deux parties ou sous-parties.}
%
%
%
%\begin{experience}
%	Expérience pour illustrer
%\end{experience}
%
%\begin{attention}
%	Erreur faite souvent ou point sur lequel insister.
%\end{attention}
\Chapter{Instruments d'optique (hors microscopes)}

\nivpre{L3}{
 \begin{itemize}
  \item optique géométrique
  \item diffraction
 \end{itemize}
}

\lemessage{Un instrument optique s'utilise dans certaines conditions pour lesquelles il a été conçu. Il y a toujours des limitations, des compromis.}

\biblio{}

\section*{Introduction}
Les instruments d'optique sont omniprésents dans la vie quotidienne, par exemple pour la plupart d'entre nous possédons des lunettes. Mais il y a d’autres applications en physique plus fondamentale comme les télescopes. Mais aussi pour sonder la matière mais qu'on étudiera pas ici, les microscopes.
\marginpar{\cite{Optique_houard}}

Autres biblio possibles \cite{Optique_hecht}, \cite{Optique_perez}


\section{Appareil photo à focale fixe}
\subsection{Rappels sur les grandissements}
\subsection{Tirage de l'objectif et mise au point}
Def ; mise au point : action qui consiste à rendre l'image nette
\subsection{Profondeur et distance hyperfocale}

\section{Lunette astronomique}
\subsection{Caractéristiques}
Lunette : système afocal : objet à l'infini renvoi une image à l'infini
\subsection{Cercle oculaire}
Def ; image de l'objectif par l'oculaire, endroit où tous les rayons issus de l'objectif par loculaire se regroupent
Contient toute l'information sur l'objet, c’est l’endroit où l'on doit placer son œil

\section{Limitations}
\subsection{Limite de résolution}
\subsection{Aberrations}
chromatique : verre=milieu dispersif, utilisation 
géométriques : coma, astigmatisme et courbure de champ, distorsion

%\begin{ecran}
%	Contenu affiché sur diapositives
%\end{ecran}
%

%
%\transition{Belle transition entre deux parties ou sous-parties.}
%
%
%
%\begin{experience}
%	Expérience pour illustrer
%\end{experience}
%
%\begin{attention}
%	Erreur faite souvent ou point sur lequel insister.
%\end{attention}

\section*{Conclusion}


\begin{remarques}
	notion du grain du récepteur : pixel.. \\
	résolution \\
	CCD ou CMOS \\
	photo couleur : 3 pixels RGB, pixels peuvent être superposés  \\
	appareil photo infos : N, tirage et temps d'exposition \\
	téléobjectif : plusieurs lentilles, donc plusieurs diaphragmes, donc perte de luminosité et plus d'aberrations \\
	corriger aberration chromatique : doublets achromatique flint-crown \\
	corrgier aberration géométriques : diaphragmes pour centrer la lumière sur l’axe ; aberrations sphériques on peut aussi utiliser un doublet ; aberrations de coma il faut que les lentilles soient parallèles à l'axe optique. En pratique, toujours un compromis, optimisation, simulations Monte-Carlo \\
	télescope c'est fait avec des miroirs et une lunette avec des lentilles \\
	pas d'ab chromatique avec miroir \\
	télescope : meilleure résolution ; taille 20m pour le plus grands \\
	placés hors des villes pour éviter la pollution lumineuses et au sommet des montagnes pour diminuer les perturbations atmosphériques ; puis optique adaptative \\
\end{remarques}

\Chapter{Spectroscopies}

\nivpre{CPGE}{
 \begin{itemize}
  \item 
 \end{itemize}
}

\lemessage{.}

\biblio{}

\section*{Introduction}
\section{Étude d'un spectre discret d'émission}
\subsection{Réseau}
\subsection{Dispositif expérimental}
\subsection{Limites}
\section{Spectroscopie interférentielle}
\subsection{Michelson}
\subsection{Finesse d'une raie}
\subsection{Spectroscopie par transformée de Fourier}

\section{Spectroscopie d'un spectre d'absorption}
\subsection{Phénomène mis en jeu}
\subsection{Spectres d'absorption}
mode de vibration, de rotation
\section*{Conclusion}

Autres : RMN (radiofréquences), spectroscopie de masse, spectroscopie électronique (LEEDS)


\Chapter{Lois de l'optique géométrique}

\nivpre{CPGE}{
 \begin{itemize}
  \item A
 \end{itemize}
}

\lemessage{Principes fondateurs de l'optique géométrique : pierres de base pour toute la conception optique, microscopie, télescopes et lunettes... Négliger interférences et diffraction : longueur à partir du cm. On en dégage des lois simples.}

\biblio{}

\section*{Introduction}
\section{Réflexion et réfraction}
\subsection{Lois de Snell-Descartes}
\subsection{Principe de Fermat}
\section{Systèmes optiques}
\subsection{Conditions de Gauss}
\subsection{Lentilles}
\section{Applications}
\subsection{Prisme}
\subsection{Mirages}
\subsection{Arc-en-ciel}
\section*{Conclusion}

\Chapter{Production et analyse de la lumière polarisée}

\nivpre{L3}{
 \begin{itemize}
  \item aspect ondulatoire de la lumière
  \item physique des ondes
  \item polarisation rectiligne et loi de Malus
 \end{itemize}
}

\lemessage{Lame quart d'onde et lame demi-onde au programme de PC.}

\biblio{}

\section*{Introduction}
limites de l'optique géométrique \\
interférences et diffraction : on a mis de côté l'aspect vectoriel de la lumière \\
on s'y intéresse ici : description, production et analyse \\

\marginpar{réf. \cite{ToutenunPC_2016},\cite{hecht},\cite{huard_pola}}

\section{Production de lumière polarisée}
\subsection{Etats de polarisation de la lumière}
Onde plane progressive harmonique OPPH \\
Expression du champ électrique \\
déphasage x,y dépend de t en générale : émission thermique, lampe à incandescence est aléatoire par exemple ; pour laser : déphasage constant \\
on a alors : polarisation rectiligne, circulaire et elliptique \\
vidéo Youtube avec propagation \\

\subsection{Production de lumière polarisée par absorption}
absorption : polariseur \\
dichroïque \\
visible : absorption par charges accélérées dans le matériau (en fait il y a peu de transmission, bcp de réflexion) \\


\subsection{Production par diffusion}
rayonnement dipolaire \\

\subsection{Production par réflexion vitreuse}
réflexion sur un dioptre \\
coefficients de Fresnel \\
angles de Brewster en polarisation TM : $\tan i_B = \frac{n_1}{n_2}$ 
exemple pour le verre : $i_b=56^o$ \\



\section{Analyse d'une lumière polarisée}


\subsection{Biréfringence}
Milieu isotrope : toutes directions ont même indice \\
Milieu biréfringent : définition, milieu qui possède deux indices de réfraction \\
soit uniaxe (no,no,ne), soit biaxe (no,ne,ne -- 2 axes optiques) \\
ex : le quartz est un milieu uniaxe \\
ie indice ordinaire et indice extraordinaire \\
ce qui nous intéresse ici : lame taillée parallèlement à axe extraordinaire \\
axes neutres, bien définir les axes \\
calcul $Delta\varphi$ \\
Lame quart d'onde : $\delta=\lambda/4$

\subsection{Action d'une lame quart d'onde}
incident rectiligne \\
cas général : polarisation elliptique \\
si $\alpha=0(\pi/2)$ pola rectiligne, si $\alpha=\pi/4(\pi/2)$ circulaire \\
incident circulaire : après lame pola rectiligne \\
incident elliptique : pola rectiligne

\section*{Conclusion}
mind map disjonction des cas \\
application : lunettes de soleil, ailleurs dans le spectre électromagnétique : antennes radio \\


\begin{remarques} \begin{itemize} 
\item la polarisation intervient dans interférences
\item onde plane : la surface d'onde est plane 
\item progressive : elle se propage dans une direction
\item on peut avoir des ondes stationnaires : pas de propagation apparente (compensation dues à conditions aux limites)
\item harmonique : une longueur d'onde = monochromatique
\item source idéale, laser s'en rapproche le plus
\item une source dans la pièce, peut on considérer onde plane ? non, quantifier avec atténuation, l'amplitude décroit en A/r
\item onde plane pour source éloignée : il faut $r>lambda$ et $r>a$
\item k E B trièdre : toujours vrai ? non : c'est Poynting, E, H ; k, D, B ; k et E ne sont pas orthogonaux dans un milieux avec des charges
\item elliptique droite ou gauche ? si on prend dans le plan $z=0$, on regarde le déplacement avec le temps, composantes $E_{0,x}$ et $E_{0,y} \cos (\varphi)$ ; $\varphi$ entre 0 et $\pi$ : gauche ; entre 0 et $-\pi$ : droite
\item polarisation par diffusion : dipôle oscille, pas d'émission dans l'axe du dipôle, il n'y a pas de composante orthoradiale 
\item réflexion vitreuse dans le cadre de programme PC, diffusion aussi
\item axe neutre d'une lame : une polarisation rectiligne reste rectiligne
\item lame à retard : définie à une longueur d'onde donnée
\item $\Delta\varphi=\frac{2\pi}{\lambda} \Delta n \, e$ : plus $\Delta n \, e$ est grand, moins on est sensible à $\lambda$, tant que multiple impair de $\frac{\lambda}{4}$
\item voir animation Arnaud + code Python
\item comment dépolariser la lumière : sur un spectre large une $\lambda/4$ avec une grande épaisseur ou sur un laser activité optique différente spatialement (prisme de Wollaston, Babinet)
\item $\frac{\lambda}{2}$ est intéressante aussi : rotation d'une polarisation rectiligne 
\item Brewster astucieusement : un dipôle dans la matière est excité, il n'y a pas de réémission dans l'axe
\item uniaxe positif : $n_e>n_0$, axe lent : axe extraordinaire (vitesse plus faible) = AO (vitesse radiale e<o)
\item Negatif : ne<no
\item Les axes optiques sont les intersections des nappes ordinaire et extraordinaires (apparaît comme milieu isotrope sur ces axes)
\item Biaxe : 2 AO ; Uniaxe : 1AO
\item prisme de Wollaston
\end{itemize} \end{remarques}
\Chapter{Optique de Fourier}

\nivpre{L3}{
 \begin{itemize}
  \item 
 \end{itemize}
}

\lemessage{.}

\biblio{}

\section*{Introduction}



\marginpar{réf. \cite{}}

\section{}
\subsection{}
\subsection{}
\section{}
\subsection{}
\subsection{}
\section{}
\subsection{}
\subsection{}


\section*{Conclusion}



\begin{remarques} \begin{itemize} 
\item 
\item 
\item 
\item 
\item 
\end{itemize} \end{remarques}
\include{LPn17}
\include{LPn24}
\Chapter{Optique de Fourier}

\nivpre{L3}{
 \begin{itemize}
  \item 
 \end{itemize}
}

\lemessage{.}

\biblio{}

\section*{Introduction}



\marginpar{réf. \cite{}}

\section{}
\subsection{}
\subsection{}
\section{}
\subsection{}
\subsection{}
\section{}
\subsection{}
\subsection{}


\section*{Conclusion}



\begin{remarques} \begin{itemize} 
\item 
\item 
\item 
\item 
\item 
\end{itemize} \end{remarques}

%\part{Quantique}

\include{Q1}
\include{Q2}
\include{Q3}
\include{Q4}
\Chapter{Modèles de l'atome}

\nivpre{L3}{
 \begin{itemize}
  \item mécanique du point
  \item équation de Schrödinger
 \end{itemize}
}

\lemessage{Depuis 2500 ans l'homme s'attache à comprendre la structure de la matière. Les développements de la mécanique quantique au cours du 20ème siècle ont permis d'établir le modèle d'aujourd'hui.}

\biblio{}

\section*{Introduction}
\section{Approches antérieures}
\subsection{De l'Antiquité aux modèles semi-classiques}
\subsection{Modèle de Bohr}
\subsection{Conséquences et limites}
\section{Modèle actuel}
\subsection{Équation de Schrödinger}
\subsection{Introduction aux nombres quantiques}
\subsection{Spin et structure de l'atome}
\section*{Conclusion}


\Chapter{Expérience de Stern-Gerlach et conséquences}

\nivpre{L3}{
 \begin{itemize}
  \item théorie quantique du moment cinétique
 \end{itemize}
}

\lemessage{.}

\biblio{}

\section*{Introduction}
passsage classique à quantique au 20ème siècle \\
c'est une expérience décisive dans cette théorie : aspect quantique des atomes et découverte du spin de l'électron  \\
à l'époque on partait du modèle de Bohr \\

%\marginpar{réf. \cite{}}

\section{Modèle de Bohr}
\subsection{Rapport gyromagnétique}
électron en trajectoire circulaire autour d'un noyau \\
boucle de courant $I$, calcul du moment magnétique (analogie spire de courant) : $\vec{M}=I \vec{S}=...=\frac{erv}{2} \vec{u_z} $\\
moment cinétique : $\vec{L}=\vec{r} \wedge \vec{p}=-m_e r v \vec{u_z}$ \\
$\vec{M}=\gamma \vec{L}$ \\
Rapport gyromagnétique : $\gamma=-\frac{e}{2 m_e}$ \\

\subsection{Précession de Larmor}
Champ magnétique $\vec{B_\mathrm{ext}}$ \\
Théorème du moment cinétique, système d'équations couplées x, y, on complexifie (x partie réelle et y partie imaginaire) \\
$L=L_x + i L_y $ donc 
$\frac{\ddroit L}{\ddroit t}=-i \gamma B_0 L$ \\
$L=L_0 exp(-i \gamma B_0 t)$ \\
pulsation de Larmor : $\omega_L = -\gamma B_0$ \\
schéma précession $\vec{L}$ autour de l'axe du champ magnétique \\

\section{Expérience de Stern et Gerlach}
\subsection{Description}
Schéma expériences \\
récipient chauffé (four) permet d'injecter atomes d'argent \\
une fente permet d'avoir seulement ceux dans une direction ($x$) \\
aimant Nord/Sud de longueur $l$, situé à distance $D$ de l'écran \\
particule vont être déviées par champ magnétique \\
champ B inhomogène suivant $z$ grâce à structure de l'aimant (sud pointe, nord trou) \\
si B était constant, le champ serait nul car $\vec{F}=\left( \vec{M} \vec{\mathrm{grad}} \right) \vec{B}$ \\
invariance par translation  \\
$F_z=M_z \frac{\partial B_z}{\partial z}$ \\
PFD dans l'entrefer : $x=v t$ ... \\
$Z=D \tan \alpha =\frac{D l}{ m_{Ag} v^2} M_z \frac{\partial B_z}{\partial z} $   \\

\href{https://youtu.be/8wS4IOzAhFA}{Video Youtube quantique Stern et Gerlach}

\subsection{Résultats}
$B_0=0$ : gaussienne centrée en $z=0$ \\
$B_0 \neq 0$ : prévision classique réponse plate, constante, pas d'angle favorisé, fonction porte \\
$B_0 \neq 0$ : observations 2 taches symétriques par rapport à $z=0$ \\


\section{Conséquences}
\subsection{Quantification du moment cinétique} 
moment cinétique total, moment cinétique orbital, moment cinétique intrinsèque \\
$\vec{J}=\vec{L}+\vec{S}$ \\
il existe un un vecteur $\vec{u_z}$ tel que $J_z$ et $\vec{J}^2$ commutent \\
propriétés des moments cinétiques, rappel nombres quantiques, $j$ et $m$ sont entiers ou demi-entiers, certaines transitions autorisées \\

\subsection{Retour sur l'expérience}
deux taches : 2 valeurs de $m$ possibles \\
$J=\frac{1}{2}$ et $m=\pm \frac{1}{2}$ \\
$M=\pm\mu_B$ où $\mu_B=\frac{e \hbar}{2 m_e}=9.274 \, 10^{-24} A.m^2 $ : quantum de moment magnétique \\
facteur de Landé : $g_L=\frac{\mu_B}{\gamma \hbar}=2$ \\

\subsection{Application à l'effet Zeeman}
$E_p (m)=-\vec{M} . \vec{B}=g\mu_B m B_0 $ \\
$E=E_0+ g \mu_B m B_0$ \\
$2J+1$ états \\
sous-niveaux Zeeman $\Delta E = E(m+1) - E(m)= g \mu_B B_0 \propto B_0$\\
applications : résonance magnétique nucléaire (RMN) du proton dans IRM \\

\section*{Conclusion}
On attribue structure probabiliste de l'atome et existence du spin \\


\begin{remarques} \begin{itemize} 
\item meilleur plan : commencer par les résultats de l'expérience ?
\item avoir les dates en tête
\item pourquoi atomes d'argent ? car gros atome à symétrie de révolution + 1 électron célibataire non apparié
\item pas un électron seul car effet cyclotron domine 
\item attention aux lignes de champ dans électro-aimant
\item électron : facteur de Landé 2 et spin demi-entier 
\item théorie classique : continuum ; vs quantification spatiale du moment cinétique (voir schéma poly Jean Hare)
\item parler de Pauli et des matrices
\item paquet d'onde s'élargit par diffraction ; en vrai c'est en forme d'œil car atomes passent pas les côtés mais y a bien deux directions privilégiées
\item axe de quantification arbitraire car symétrie cylindrique ; quand on applique le champ $\vec{B}$, on brise la symétrie
\item voir vidéo Youtube MIT ? 
\end{itemize} \end{remarques}
\Chapter{Equation de Schrödinger et applications}

\nivpre{CPGE}{
 \begin{itemize}
  \item optique ondulatoire
  \item 
  \item 
  \item 
 \end{itemize}
}

\lemessage{.}

\biblio{}

\section*{Introduction}
on a vu relation onde - corpuscule, longueur d'onde de De Broglie \\
expérience introductive : diffraction des électrons \\
accélération des électrons par tension  \\
sur graphite, diffraction comme un réseau en optique \\
matière est une onde ! \\
comment décrire ces états ? \\

Slide dualité onde-corpuscule : relation de De Broglie $\lambda_{db}=\frac{h}{p}$ ; pour des électrons accélérés : $\lambda_{db}=\frac{rd}{l}$ \\

%\marginpar{réf. \cite{ToutenunPC_2016},\cite{hecht},\cite{huard_pola}}

\section{État et évolution d'une particule}
\subsection{Fonction d'onde}
$\psi (M,t)$, valeurs complexes, interprétation : amplitude de probabilité de trouver la particule à un instant donné \\
Dans un volume $dV$ centré sur $M$ à l'instant $t$, la probabilité est : $dP=\lvert \psi(M,t) \lvert ^2  dV$ \\
caractère probabiliste \\
propriété importante, loi de conservation : intégré sur tout l'espace, on obtient $1$ puisqu'on sait que cette particule existe \\
 Comment étudier son évolution dans le temps ?  \\ 
\subsection{Équation de Schrödinger}
1925 \\
on étudie ici à 1D \\
$\psi=\psi_0 e^{i\omega t}$ \\
$E=\hbar \omega=\frac{p^2}{2m}+V$ ; $c=\lambda_{db} \nu$ \\
on part de l'équation de propagation et on démontre Schrödinger : $E \psi=-\frac{\hbar^2}{2m} \frac{\partial^2  \psi}{\partial x^2} +V(x) \psi$ \\
forme générale avec opérateur (pas démontrée ici) : $i\hbar \frac{\partial \psi}{\partial t}=-\frac{\hbar^2 }{2 m} \frac{\partial^2  \psi}{\partial x^2} + V(x) \psi $ \\
connaissance de l'évolution de la fonction d'onde\\
équation linéaire : si on a deux solutions, toute combinaison linéaire est alors solution \\

\section{Solutions}
\subsection{Cas libre}
$V=0$, on cherche des solutions stationnaires $\psi=\phi(x) f(t)$ \\
on les appelle états stationnaires \\
démonstration à partir de normalisation dans l'espace, et $f(t)=e^{-i\omega t}$ \\
$\psi ''+\frac{2m\omega}{\hbar} \psi$ =0 \\
$K=\sqrt{\frac{2m\omega}{\hbar}}$ : pulsation spatiale \\

\subsection{Cas d'une barrière de potentiel}
3 zones \\
énergie supérieure ou inférieure à $V_0$... \\

\section{Applications}

\subsection{Effet tunnel}

courant de probabilité (simplifié pour CPGE): $\vec{J}=\frac{\hbar \vec{k}}{m} \lvert \psi \lvert ^2$ \\

slide : radioactivité $\alpha$, He franchit une barrière de potentiel par effet tunnel \\
microscope à effet tunnel : sonder la matière à l'échelle nanométrique et manipulation d'atomes \\
\href{https://youtu.be/oSCX78-8-q0}{vidéo réalisée image par image}



\section*{Conclusion}
ne prend pas en compte effets relativistes 

\begin{remarques} 
Améliorations : 
\begin{itemize} 
\item plutôt introduire l'équation de Schrodinger à partir de l'énergie cinétique et énergie potentielle 
\item faire une seule particule
\item qqch de complètement différent de classique
\item probabilité de présence ? pas exactement, c'est qqch qui nous permet de déterminer la proba
\item champ de l'espace et du temps défini en tout point
\item aspect dynamique pour évolution temporelle : on a longueur d'onde de de broglie, vecteur d'onde, le même que Fourier ; consiste juste à établir relation de dispersion
\item on a psi à un moment donné : on intègre, alors on l'a à n'importe quel instant
\item on s'intéresse aux modes propres car solutions
\item la mesure est un phénomène probabiliste
\item hypothèses : $\psi$ et $\frac{d \psi}{dt}$ sont continus
\item puits de potentiel est plus intéressant pour les aspects de normalisation
\end{itemize}

Questions :
\begin{itemize} 
\item évolution d'un système quantique, tout système quantique ?   non
\item particules élémentaires : photon, boson de Higgs, quark, électrons... 
\item ça marche pour le photon ? non, pas de charge conservée, exemple corps noir : on crée des photons, oscillateurs
\item c'est quoi le modèle standard ? 
\item y a t'il des théories quantiques qui ne sont pas de la mécanique quantique ?
\item théorie quantique des champs ? système à nombres de particules variables, qui s'ajustent
\item au sens strict, la mécanique quantique relativiste n'existe pas 
\item passage à la mécanique classique ? théorème d'Ehrenfest ou approximation WKB avec paquets d'onde
\end{itemize} \end{remarques}


%\part{Matière, solide}

\include{S1}
\include{S2}
\include{S3}
\Chapter{Application des semiconducteurs à l'électronique et/ou l'optique }

\nivpre{Licence}{
 \begin{itemize}
  \item physique statistique : distribution de Fermi-Dirac
   \item caractéristique d'une photodiode
 \end{itemize}
}

\lemessage{Énormément utilisé dans les capteurs : on module la conduction pour faire plein de capteurs différents.}

\biblio{}

\section*{Introduction}
plein d'applications, toute l'électronique qu'on utilise aujourd'hui \\

%\marginpar{réf. \cite{ToutenunPC_2016},\cite{hecht},\cite{huard_pola}}

\section{Propriétés des semiconducteurs}
\subsection{Bande de valence, bande de conduction et énergie de gap}
Structure de bande \\
pour un semiconduteur énergie de gap inférieure à $5eV$ \\
pour un isolant énergie de gap supérieure à $5eV$ \\
A $T=0$, tous les électrons sont dans la bande de valence \\
A $T>0$, certains électrons passent dans la bande de conduction \\
densité de porteurs : $n_e$ électrons présents dans le BC, $n_h$ trous dans la BV \\
en pratique $n_e=N_e e^{\left( \frac{E_g-E_c}{k_B T}\right)}$  et  $n_h=N_ h e^{\left( \frac{E_g-E_c}{k_B T}\right)}$ \\


\subsection{Dopage}
exemple : on part d'un silicium intrinsèque (4 électrons de valence) \\
on remplace des atomes de silicium par des atomes d'arsenic (5 électrons de valence, très peu)  \\

change énergie de Fermi \\
dopé p : niveau de Fermi plus proche du haut de la bande de valence \\
dopé n : niveau de Fermi plus proche du bas de la bande de conduction \\

2 types de dopage : n $\rightarrow$ on ajoute des électrons, p $\rightarrow$ on ajoute des trous \\


\section{La jonction PN}
\subsection{Présentation du phénomène}
accoller matériau dopé p et matériau dopé p \\
les électrons s'équilibrent sur la zone à la limite des 2 \\
vidéo Youtube : formation and properties of junction diode   \\
zone de déplétion : il y a ni trou, ni électron $\rightarrow$ la circulation est bloquée \\
appariation d'un champ dans la ZCE, ie différence de potentiel : $V_0 = E_F^N -E_F^P = \frac{k_B T}{e} \log{\frac{N_a N_d }{n_i^2}}$ \\
dans un circuit, on va donc polariser la jonction \\

\subsection{Polarisation de la jonction}
on applique une différence de potentiel à la jonction PN \\
on change donc la différence de potentiel : $V_0'=V_0-V_a$, supérieure ou inférieure à $V_0$ \\
 caractéristique d'une diode \\
 illustration expérimentale avec une photodiode en changeant la résistance \\

\section{Applications}
\subsection{Application à la photodétection de lumière : la photodiode}
$V_a<0$ : polarisation en inverse \\
comment ça marche ? \\
on récupère les électrons "produits" du côté zone p \\
caractéristique : un photon excite un électron : création paire électron/trou \\
définition courant photonique \\

\subsection{Application à l'émission de lumière : la DEL (ou LED)}
$V_a>0$ : polarisation directe \\
électrons injectés du côté zone n \\
on peut faire des lasers \\
vidéo fonctionnement \\

\subsection{Application à l'électronique : le redresseur}
signal sinusoïdal envoyé à une diode avec une résistance de protection \\
lien avec signe diode, redressement \\

\section*{Conclusion}
on a présenté le cas le plus simple : jonction PN \\
on peut aussi faire PNP et NPN : amplificateur de puissance  \\
on peut faire thermistance car passage des électrons est régit par une distribution de Fermi-Dirac, résistance va dépendre de température \\
présents partout : téléphones, caméra, ordinateur...\\

\begin{remarques} \begin{itemize} 
\item donner des ordres de grandeur de gap pour différents matériaux
\item intérêt de la photodiode PIN : augmenter artificiellement la taille de la ZCE. Ainsi, la majorité des photons y est absorbée. De plus, cette région intrinsèque étant pure ($99.99 \%$ pour le silicium), la vitesse des porteurs y est significativement augmentée. En effet, ces derniers n’y subissent que très peu de collisions du fait de cette absence d’impureté
\item bandes de valence, conduction et interdite : en physique des solides, on résout équation de Schrödinger, potentiel périodique du cristal ; la fonction d'onde d'un électron dans le cristal
\item théorème de Bloch
\item énergie en fonction du vecteur d'onde k 
\item important à mettre : diagramme simplifique en fonction de k, paraboles
\item gap direct/gap indirect
\item pourquoi quand il y a des électrons dans bande de conduction il y a un courant électrique ? 
\item pourquoi parabolique ? car écart à électron libre $\frac{\hbar^2 k^2 }{2m^* }$
\item dissymétrie en fonction de k : 
\item énergie de Fermi d'un conducteur, peuplement des niveaux d'énergie
\item intrinsèque : non dopé ; utilisation : filtre optique (GaP), photorésistance (CdS), thermorésistance, sonde à effet Hall
\item $E_g = \frac{h c}{\lambda_g} = \frac{ 1.24 \left[ eV \right] }{ \lambda \left[ \mu m \right] }$
\item extrinsèque : dopé ; utilisation : transistor 
\item caractéristique de la photodiode : exponentielle ; zone génératrice, photovoltaïque : photopile ; si balayée trop rapidement (?), on observe des effets capacitifs
\item $V_0=0,7V$ pour le silicium
\item faire un courant alternatif : deux diodes en parallèle sens opposés
\item ordre de grandeur des températures de Fermi des métaux, conducteurs : $10^4 K$
\end{itemize} \end{remarques}
\Chapter{Fermions, bosons, illustrations}

\nivpre{Licence}{
 \begin{itemize}
  \item distribution canonique et grand canonique
   \item traitement microcanonique du gaz parfait
   \item postulat de symétrisation mécanique quantique
   \item notion de spin
 \end{itemize}
}

\lemessage{.}

\biblio{}

\section*{Introduction}
aspects quantique du modèle microcanonique du gaz parfait \\

%\marginpar{réf. \cite{ToutenunPC_2016},\cite{hecht},\cite{huard_pola}}

\section{Formalisme des statistiques quantiques}
\subsection{Rappel : résultat semi-classique}
entropie de Sackur-Tetrode \\
$S>0$ : limite diluée, $n\Lambda^3 <= 1$ \\
quantique : recouvrement fonctions d'onde \\

\subsection{Formalisme des statistiques quantiques}
Pour un système de N particules identiques, les seuls états physiques sont antisymétriques car symétriques par échange de particules.  \\
Boson, fonction d'onde symétrique par échange de deux particules, exemples : photon, He4 \\
Fermion, fonction d'onde antisymétrique par échange de deux particules, exemples : électrons, proton \\
Le spin des fermions est demi-entier \\
Le spin des bosons est entier \\
exmple : 2 bosons dans 2 états a et b, écriture de l'état, + entre les deux fonctions d'ondes  \\
2 fermions dans 2 états a et b, écriture avec un - entre les deux fonctions d'ondes \\
Application : principe d'exclusion de Pauli, deux fermions ne peuvent pas être dna sle même état quantique \\

\subsection{Factorisation de la fonction de partition (ensemble grand-canonique)}
T et $\mu$ fixés, N particules \\
$\Xi=\Sigma e^{\beta \left( E_l - \mu N_\lambda \right) }=\Sigma \Pi ...$  \\

formule de $<N_\lambda>$ en fonction de la dérivée de $ln ( \xi_\lambda )$ \\


\section{Distribution de Fermi-Dirac et Bose-Einstein}
\subsection{Stat de Bose-Einstein}
boson : $N_\lambda=0, 1,..., \infty$ \\
calcul de $\xi_\lambda$ \\
\subsection{Stat de Fermi-Dirac}
fermion : $N_\lambda=0$ ou $1$ \\
calcul de $\xi_\lambda$ \\

tracé des deux caractéristiques en fonction de $\beta(E_\lambda - \mu)$ : tendent va la même valeur \\
tendent vers la statistique de Maxwell-Boltzmann, expression \\

\section{Applications}
\subsection{Rayonnement du corps noir}
une boîte cubique \\
conditions aux limites périodiques \\
quantification des vecteurs d'onde \\
calcul calotte sphérique \\
ne pas oublier les 2 états de polarisation \\
densité spectrale de modes \\
on trouve la loi de Planck \\
loi de Wien \\
tracé pour différentes températures \\

\subsection{Comportement des électrons libres d'un métal}
équation de Schrödinger \\
densité d'énergie, calcul, en $E^\frac{1}{2}$ \\
à $T=0K$, FD devient fonction de Heavyside \\
$N=A E_F ^\frac{3}{2}$, expression énergie de Fermi : de quelques eV à quelques 10eV \\

\section*{Conclusion}


\begin{remarques} \begin{itemize} 
\item bien définir les fonctions de partition, tous les termes
\item Sackur-Tetrode dans micro-canonique, en fait ici on a donné son expression dans l'ensemble canonique (fonction de N, V, T)
\item théorie des champs : théorème spin statistique (hors programme)
\item fonction d'onde : N fermions ou N bosons, déterminant de Slater
\item loi de Planck : équilibre thermodynamique
\item thermostat : parois
\item loi du déplacement de Wien 
\item condensat de Bose : tous les bosons sont dans le même état, dans l'espace des k, refroidissement avec champ magnétique ou laser, la température est déterminée en fittant la "queue de la distribution" de population
\item sphère de Fermi : sphère en vecteur d'onde qui correspond à l'énergie de Fermi
\item masse de Chandrasekhar : masse limite au-delà de laquelle il y a effondrement gravitationnel, devient étoile à neutron
\item supraconducteur : paire de Cooper
\end{itemize} \end{remarques}



\end{document}

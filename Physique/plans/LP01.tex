\Chapter{Gravitation}

\nivpre{CPGE}{
 \begin{itemize}
  \item Mécanique de point
  \item Théorèmes généraux mécanique
  \item Électrostatique
 \end{itemize}
}

\lemessage{Attention demo trajectoires coniques n'est plus au programme de CPGE depuis 2014, nécessité de placer en L3 ou admettre le résultat sans trop analyser mathématiquement, mais les trajectoires sont à connaître sans les formules géométriques.}

\biblio{}


\section*{Introduction}
modèle géocentrique, planètes ont été identifiées : se déplacent dans le ciel nocturne \\
modèle héliocentrique : Copernic \\
\href{https://solarsystem.nasa.gov/solar-system/our-solar-system/overview/}{animation NASA}
ou
\href{http://www.planete-astronomie.com/animation-de-la-position-des-planetes.html}{animation astronomie}




\section{Interaction gravitationnelle}
\subsection{Force gravitationnelle}
mentionner troisième loi de Newton : actions réciproques
\subsection{Champ de pesanteur}
\subsection{Énergie potentielle gravitationnelle}
\section{Mouvement dans un potentiel gravitationnel}
\subsection{Position du problème}
\subsection{Potentiel effectif}
démo coniques seulement si choix L3
\subsection{Lois de Kepler}
\section{Application : vitesse de libération}
\section*{Conclusion}

%\begin{ecran}
%	Contenu affiché sur diapositives
%\end{ecran}
%
%\begin{remarques}
%	Remarque concernant le contenu
%\end{remarques}
%
%\transition{Belle transition entre deux parties ou sous-parties.}
%
%\begin{itemize}
%	\item Calcul, définition, etc. : le contenu de la leçon 
%\marginpar{réf. \cite{ToutenunPCSI_2003}}
%	\item \begin{equation*}
%		\rot{E} = - \pdv{\vv{B}}{t} \qq{et} \oiint_\mathcal{S} \vv{E} \cdot \vv{\dd{S}} = \frac{Q_\text{int}}{\varepsilon_0} \,.
%	\end{equation*}
%\end{itemize}
%
%\begin{experience}
%	Expérience pour illustrer
%\end{experience}
%
%\begin{attention}
%	Erreur faite souvent ou point sur lequel insister.
%\end{attention}


\begin{remarques} \begin{itemize}
\item unité de G ? mesure de G ? pendule de torsion, horloge atomique
\item masses ponctuelles ?
\item traditionnellement ce que l'on appelle pesanteur sur Terre : on inclut force (centrifuge) d'inertie d'entrainement (référentiel terrestre non galiléen, Terre tourne sur elle-même) et force de marée 
\item analogue pour champ magnétique B dans gravitation ? non
\item modèle de la Terre creuse : roman de Jules Verne... alors par de gravitation à l'intérieur (Gauss, masse nulle entourée donc force nulle)
\item rotG ? nul
\item pas de masse négative : force gravitationnelle d'un anti-atome d'hydrogène, interagissent de la même manière par interaction gravitationnelle ? en cours de recherche
\item on peut négliger variations gravitation à la surface de la Terre, mesure ? gravimétrie, applications économiques : recherche pétrole, mesure différence de densité
\item mesure de la chute libre par Galilée : tour de Pise, pas de chronomètre, mesure du temps à l'aide de son pouls
\item détecter une planète : voir changement de trajectoires, clignotement quand passent devant leur étoile, lunette gravitationnelle, méthode des transits 
\end{itemize} \end{remarques}
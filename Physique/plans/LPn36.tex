\Chapter{Entropie en physique statistique et thermodynamique}

\nivpre{L3}{
 \begin{itemize}
  \item 
  \item 
  \item 
 \end{itemize}
}

\lemessage{.}

\biblio{}

\section*{Introduction}

\section{}
\section{}
\section{}


\section*{Conclusion}



\begin{remarques} \begin{itemize} 
\item introduit en 1865 par Rudolf Clausius, un terme qui fait référence à la fois à l'énergie et à la transformation
\item l'entropie d'un corps noir est proportionnelle à son aire (et colossale), renseigne sur quantité d'information que renferme le trou noir ; lorsqu'un objet tombe dedans, le trou noir absorbe son entropie, ce qui permet à l'univers de pas avoir une entropie qui diminue
\item plasmas : convergent vers équilibre sans changement d'entropie, réversible grâce à interaction à distance (champ électrique) ; Landau prétendait que même sans collision, le plasma se rapprocherait de l’équilibre suite à une diminution du champ électrique (pou équation de Vlasov-Poisson linéarisé, Villani l'a démontré avec non-linéaire)
\item 
\item 
\item 
\item 
\item 
\item 
\end{itemize} \end{remarques}
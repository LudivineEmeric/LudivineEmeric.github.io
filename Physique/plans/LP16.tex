\Chapter{Microscopies optiques}

\nivpre{L3}{
 \begin{itemize}
\item optique géométrique
\item diffraction (critère de Rayleigh)
\item fluorescence
 \end{itemize}
}

\lemessage{Message important à faire passer lors de la leçon.}

\biblio{}

\section*{Introduction}
Microscope : appareil permettant d’étudier des détails microscopiques à une échelle macroscopique.
Petit historique : utilisation systématique de lentilles grossissantes à la fin du XVIe siècle. On veut : bon grossissement, bonne résolution, image fidèle (pas de défauts).

\section{Microscope classique}
\subsection{Dispositif}
\marginpar{\cite{Optique_houard} p.154}
\marginpar{\cite{app_miscroscope}}
\subsection{Grossissement commercial}

\subsection{Limites}
résolution : diffraction, critère de Rayleigh \\
aberrations des lentilles : correction des aberrations chromatiques avec verre de champ et verre d’oeil  \marginpar{\cite{Optique_houard} p.161}

\transition{Biologie échantillons transparents non colorés : compliqué.}

\section{Microscopie par contraste de phase}

\marginpar{\cite{constrastephase_kastler}}
\marginpar{\cite{phasecontrast_microscopy}}
\href{http://toutestquantique.fr/champ-sombre-et-contraste/}{Vidéo Youtube Tout est quantique sur le champ sombre et contraste}


Objet de phase : $s_0=V_0 \exp(i\omega t)$ et $s=s_0 \exp(i\varphi)$
\begin{itemize}
\item Strioscopie : on coupe $s_0$, donc on obtient $I=I_0 \varphi^2$. Problèmes : contraste très faible
\item Contraste de phase : $s_0$ retardé de $-\pi/2$ (lame de verre d’épaisseur $n\lambda+\lambda/4$ par exemple), on
obtient $I \approx I_0(1-2\varphi)$. Alors $C=\abs{2\varphi}$. Prix Nobel 1953 Zernike
\end{itemize}

Avantages : méthode non destructive, non intrusive. On peut observer du vivant!  \\

\transition{Observer parties spécifiques ou échantillon épais : 3D}

\section{Microscope confocale laser à fluorescence}
\marginpar{\cite{cohard_confocale}}
\marginpar{\cite{fluorescence_microscopy}}
Fluorescence : fluorophores dans l’échantillon (qui se fixent spécifiquement).  \\
Confocale laser : image point par point \\
Importance du miroir dichroïque. C’est là l’intérêt de la fluorescence : on peut facilement éliminer la lumière diffusée. \\


\section*{Conclusion}
méthodes non optiques : STM, AFM, MEB, TEM \\
résolution nanométrique et sub


\begin{remarques} Détails sur le microscope à contraste de phase
\begin{itemize}
\item Éclairage de Köhler : il permet d’éclairer l’échantillon de manière uniforme, et de ne pas avoir le filament superposé à l’échantillon (ce que l’on fait habituellement pour avoir la meilleur luminosité). Solution : faire l’image de la source sur le diaphragme d’ouverture du condenseur (ici le "condenser annulus"), qui est dans le plan focal objet du condenseur.
\item L'anneau de phase est placé dans le plan focal image de l’objectif, et c’est bien le plan de Fourier (on a bien un éclairage parallèle sur l’échantillon). Ce système conjugue ainsi le "condenser annulus" et le "phase plate". Par ailleurs l’objectif fait l’image de l’échantillon sur l’écran.
\item Pour bien comprendre le schéma, il faut voir que les traits sont des rayons (le coloriage entre traits est en fait plus perturbant qu’autre chose), et lorsqu’ils se recoupent on peut voir où est formée l’image de tel ou tel objet.
\end{itemize}
\end{remarques}


%\begin{ecran}
%	Contenu affiché sur diapositives
%\end{ecran}

%\begin{remarques}
%	Remarque concernant le contenu
%\end{remarques}

%\transition{Belle transition entre deux parties ou sous-parties.}



%\begin{experience}
%	Expérience pour illustrer
%\end{experience}

%\begin{attention}
%	Erreur faite souvent ou point sur lequel insister.
%\end{attention}
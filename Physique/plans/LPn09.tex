\Chapter{Inductances de l'électromagnétisme à l'électrocinétique}

\nivpre{CPGE}{
 \begin{itemize}
 \item théorème d'Ampère
 \item loi de Lenz, loi de Faraday
 \item bases de l'électrocinétique (lois des mailles, noeuds)
 \item utilisation de complexes en EC
 \end{itemize}
}

\lemessage{.}

\biblio{}

\section*{Introduction}
il existe des forces électromotrices induites, couplage avec le champ magnétique \\
comment ça marche dans un circuit ? cadre de validité ? \\
\marginpar{réf. \cite{Neveu_elec},\cite{ToutenunPSI_2016}, \cite{Perez_EM}, \cite{BFR_EM_4}}

\section{Rappel des lois de l'induction}
\subsection{Notions de flux et d'inductance}
flux $\Phi=...$ \\
on définit l'inductance par $L=\frac{\Phi}{i}$ \\
\subsection{Loi de Faraday}
force électromotrice induite $fem=e=-\frac{d\Phi}{dt}$ \\
alors $e=-L \frac{di}{dt}$ car L est constant \\
on retrouve loi classique de l'électrocinétique
\section{Inductance propre/mutuelle}
\subsection{Inductance propre}
Étude d'un solénoïde, théorème d'Ampère \\
$\Phi=\mu_0 \frac{N^2 S}{l}i$ \\
donc $L=\mu_0 \frac{N^2 S}{l}$ \\
ne dépend pas du temps tant que le circuit ne bouge pas, est indéformable  \\

\subsection{Inductance mutuelle}
il n'y a pas d'aimant, seulement le circuit, avec un autre : on a 2 solénoïdes \\
représentation des lignes de champ de chacun des solénoïdes \\
$M=\frac{\Phi_{1 \rightarrow 2}}{i_1}$ \\
Pour 2 solénoïdes imbriqués de même surface : $M=\mu_0 \frac{N_1 N_2}{l}$
\subsection{Passage à l'électrocinétique : schéma électrique équivalent}
pour une inductance propre : convention générateur et convention récepteur \\
la fem est définie en convention générateur \\
Pour 2 circuits couplés par inductance mutuelle : $U_1=L_1 \frac{di_1}{dt} + M \frac{di_2}{dt}$,...

\section{Utilisation en électrocinétique}
\subsection{Mesure de L par un circuit RL}
on a une petite résistance interne, que l'on néglige \\
$E=L \frac{di}{dt}+R i$ \\
réponse à un créneau, aux bornes de la résistance : $u(t)=E \left( 1-e^{-\frac{t}{\tau}} \right) $ avec $\tau = \frac{L}{R}$ \\
continu, faible variation donc passe-bas aux bornes de la résistance \\

expérience, décrément logarithmique : $0.63 E$ \\

\subsection{Mesure de M}
$u_1=L \frac{di_1}{dt}+M \frac{di_2}{dt}$ \\
$u_2=L \frac{di_2}{dt}+M \frac{di_1}{dt}$ \\
on prend une résistance très grande dans le circuit secondaire pour négliger $i_2$ \\
$\frac{u_2}{u_1}=\frac{M}{L}$


\subsection{Transformateur}
2 bobines traversées par le même champ B \\
en pratique : il y a un support (ferro) qui permet de canaliser les lignes de champ \\
rapport de transformation : $m=\frac{N_2}{N_1}=\frac{U_2}{U_1}$ \\
expérience : étude d'un transformateur 

\subsection{Circuit RL en régime sinusoïdal forcé} 
sortie aux bornes de la bobine \\
filtre passe-haut \\


\section*{Conclusion}
applications : filtrage, effet présent dans câble coaxial \\

\begin{remarques} \begin{itemize} 
\item toujours orienter les contours, lignes de champ
\item définition pulsation de coupure
\item si pas de variation de courant, on n'a pas d'induction, loi de Faraday
\item tu as fait circuit filiforme (1D), comment ça marche pour un circuit non-filiforme ? (2D)
\item solénoïde infini de taille fini : il faut que longueur soit très grande devant le rayon
\item l'auto-inductance est toujours positive car orientations sont définies comme ça
\item l'inductance propre ne dépend pas du métal, ne dépend pas de la conductivité électrique
\item l'inductance mutuelle de 1 sur 2 et de 2 sur 1, c'est pareil ?  oui, il faudrait le calculer avec Biot-et-Savart
\item inductance mutuelle : dépend de géométrie, matériau entre les 2, peut être négatif, dépend comment on oriente les circuits les uns par rapports aux autres
\item $K=\frac{M}{\sqrt{L_1 L_2}}=1$ pour bobines en influence totale (une ligne de champ passant dans une bobine passe aussi dans l'autre)
\item $abs(K)<=1$, dem : $E_{tot}=\frac{1}{2} \left( i_1 \phi_1 +i_2 \Phi_2 \right)=\frac{1}{2}\left( L_1 i_1^2  +L_2 i_2^2 + 2M i_1 i_2 \right)$, $E_{int}=M i_1 i_2$, et $E_{tot}=\frac{\left( \vec{B_1} + \vec{B_2}\right)^2}{2\mu_0} V$ donc $M^2<L_1 L_2$
\item induction de Neumann ou Lorenz ? Neumann, circuit fixe dans un champ variable
\item circuit filiforme $E=\frac{1}{2} \Phi i$
\item reconnaître un passe-bas, un passe-haut à l'allure de la réponse à un échelon
\item des formules de réciprocité...
\item montrer que champ magnétique est nul à l'extérieur : voir TD Erwan
\item voir L non contant avec géométrie variable : force d'un circuit sur lui même (tendance à s'ouvrir, s'éloigner), force créée par le solénoïde a tendance à l'écraser lui même ; action magnétique subie par le courant, force subie par une spire, règle du flux maximal : énergie potentielle associée aux forces de Laplace circuit se déplace de façon à maximiser $i \Phi$, variation positive de $L$
\item autre piste : quelle nouvelle vision peut nous donner l'électromagnétisme sur la bobine en électrocinétique ?
\end{itemize} \end{remarques}